%% Generated by Sphinx.
\def\sphinxdocclass{report}
\documentclass[letterpaper,10pt,english]{sphinxmanual}
\ifdefined\pdfpxdimen
   \let\sphinxpxdimen\pdfpxdimen\else\newdimen\sphinxpxdimen
\fi \sphinxpxdimen=49336sp\relax

\usepackage[margin=1in,marginparwidth=0.5in]{geometry}
\usepackage[utf8]{inputenc}
\ifdefined\DeclareUnicodeCharacter
  \DeclareUnicodeCharacter{00A0}{\nobreakspace}
\fi
\usepackage{cmap}
\usepackage[T1]{fontenc}
\usepackage{amsmath,amssymb,amstext}
\usepackage{babel}
\usepackage{times}
\usepackage[Sonny]{fncychap}
\usepackage{longtable}
\usepackage{sphinx}

\usepackage{multirow}
\usepackage{eqparbox}

% Include hyperref last.
\usepackage{hyperref}
% Fix anchor placement for figures with captions.
\usepackage{hypcap}% it must be loaded after hyperref.
% Set up styles of URL: it should be placed after hyperref.
\urlstyle{same}
\addto\captionsenglish{\renewcommand{\contentsname}{Contents:}}

\addto\captionsenglish{\renewcommand{\figurename}{Fig.\@ }}
\addto\captionsenglish{\renewcommand{\tablename}{Table }}
\addto\captionsenglish{\renewcommand{\literalblockname}{Listing }}

\addto\extrasenglish{\def\pageautorefname{page}}

\setcounter{tocdepth}{3}
\setcounter{secnumdepth}{3}


\title{AFLOWpi Developers Guide}
\date{Jan 30, 2017}
\release{}
\author{Andrew Supka}
\newcommand{\sphinxlogo}{}
\renewcommand{\releasename}{Release}
\makeindex

\begin{document}

\maketitle
\sphinxtableofcontents
\phantomsection\label{\detokenize{index::doc}}



\chapter{plot module}
\label{\detokenize{plot:module-plot}}\label{\detokenize{plot::doc}}\label{\detokenize{plot:plot-module}}\label{\detokenize{plot:welcome-to-aflowpi-s-documentation}}\index{plot (module)}\index{bands() (in module plot)}

\begin{fulllineitems}
\phantomsection\label{\detokenize{plot:plot.bands}}\pysiglinewithargsret{\sphinxcode{plot.}\sphinxbfcode{bands}}{\emph{calcs, yLim={[}-10, 10{]}, DOSPlot='`, runlocal=False, postfix='`, tight\_banding=False}}{}
Generates electronic band structure plots for the calculations in the dictionary of dictionaries
of calculations with the option to have a DOS or PDOS plot to accompany it.
\begin{quote}\begin{description}
\item[{Parameters}] \leavevmode
\sphinxstyleliteralstrong{calcs} (\sphinxstyleliteralemphasis{dict}) -- dictionary of dictionaries representing the set of calculations

\item[{Keyword Arguments}] \leavevmode\begin{itemize}
\item {} 
\sphinxstyleliteralstrong{yLim} (\sphinxstyleliteralemphasis{list}) -- List or tuple of two integers for max and min range in horizontal axis of DOS plot

\item {} 
\sphinxstyleliteralstrong{LSDA} (\sphinxstyleliteralemphasis{bool}) -- To plot DOS as spin polarized or not (calculation must have been done as spin polarized)

\item {} 
\sphinxstyleliteralstrong{DOSPlot} (\sphinxstyleliteralemphasis{str}) -- DOS or the PDOS plots next to the eletronic band structure plot (assuming you ran
either ppDOS or ppPDOS) (default: NONE)

\item {} 
\sphinxstyleliteralstrong{postfix} (\sphinxstyleliteralemphasis{str}) -- Postfix to the filename of the plot

\item {} 
\sphinxstyleliteralstrong{tight\_banding} (\sphinxstyleliteralemphasis{bool}) -- Whether to treat the input data as from Quantum Espresso or WanT bands.x

\end{itemize}

\item[{Returns}] \leavevmode
None

\end{description}\end{quote}

\end{fulllineitems}

\index{distortionEnergy() (in module plot)}

\begin{fulllineitems}
\phantomsection\label{\detokenize{plot:plot.distortionEnergy}}\pysiglinewithargsret{\sphinxcode{plot.}\sphinxbfcode{distortionEnergy}}{\emph{calcs1}, \emph{xaxis}, \emph{yaxis}, \emph{zaxis='Energy'}, \emph{calcs=None}, \emph{colorbarUnits=None}, \emph{titleArray=None}, \emph{plotTitle=None}, \emph{xAxisStr=None}, \emph{yAxisStr=None}, \emph{fileName='distortionEnergy.pdf'}, \emph{percentage=False}, \emph{runlocal=True}}{}
\end{fulllineitems}

\index{dos() (in module plot)}

\begin{fulllineitems}
\phantomsection\label{\detokenize{plot:plot.dos}}\pysiglinewithargsret{\sphinxcode{plot.}\sphinxbfcode{dos}}{\emph{calcs, yLim={[}-10, 10{]}, runlocal=False, postfix='`}}{}
Generates DOS plots for the calculations in the dictionary of dictionaries of calculations
\begin{quote}\begin{description}
\item[{Parameters}] \leavevmode
\sphinxstyleliteralstrong{calcs} (\sphinxstyleliteralemphasis{dict}) -- dictionary of dictionaries representing the set of calculations

\item[{Keyword Arguments}] \leavevmode\begin{itemize}
\item {} 
\sphinxstyleliteralstrong{yLim} (\sphinxstyleliteralemphasis{list}) -- List or tuple of two integers for max and min range in horizontal axis of DOS plot

\item {} 
\sphinxstyleliteralstrong{LSDA} (\sphinxstyleliteralemphasis{bool}) -- To plot DOS as spin polarized or not (calculation must have been done as spin polarized)

\item {} 
\sphinxstyleliteralstrong{runlocal} (\sphinxstyleliteralemphasis{bool}) -- Do the plotting right now or if False do it when the calculations are running

\item {} 
\sphinxstyleliteralstrong{postfix} (\sphinxstyleliteralemphasis{str}) -- Postfix to the filename of the plot

\item {} 
\sphinxstyleliteralstrong{tight\_banding} (\sphinxstyleliteralemphasis{bool}) -- Whether to treat the input data as from Quantum Espresso or WanT bands.x

\end{itemize}

\item[{Returns}] \leavevmode
None

\end{description}\end{quote}

\end{fulllineitems}

\index{grid\_plot() (in module plot)}

\begin{fulllineitems}
\phantomsection\label{\detokenize{plot:plot.grid_plot}}\pysiglinewithargsret{\sphinxcode{plot.}\sphinxbfcode{grid\_plot}}{\emph{calcs}, \emph{xaxis}, \emph{yaxis}, \emph{zaxis='Energy'}, \emph{colorbarUnits=None}, \emph{zaxis\_title=None}, \emph{plot\_title=None}, \emph{xAxisStr=None}, \emph{yAxisStr=None}, \emph{fileName='grid\_plot.pdf'}, \emph{runlocal=True}}{}
\end{fulllineitems}

\index{interpolatePlot() (in module plot)}

\begin{fulllineitems}
\phantomsection\label{\detokenize{plot:plot.interpolatePlot}}\pysiglinewithargsret{\sphinxcode{plot.}\sphinxbfcode{interpolatePlot}}{\emph{calcs}, \emph{variable1}, \emph{variable2}, \emph{zaxis='Energy'}, \emph{xaxisTitle=None}, \emph{yaxisTitle=None}, \emph{zaxisTitle=None}, \emph{title=None}, \emph{fileName='interpolatePlot.pdf'}, \emph{delta=False}, \emph{text\_min=False}, \emph{vhline\_min=False}, \emph{circle\_min=False}, \emph{delta\_min=True}, \emph{rel\_center=False}, \emph{plot\_color='jet'}, \emph{find\_min=False}}{}
Takes a list of calculations and plots the energy of the calculations as a function of two input variables
the first value is the baseline for the energy value and the energy plotted is the difference between that
energy and the other energies in the grid
\begin{quote}\begin{description}
\item[{Parameters}] \leavevmode\begin{itemize}
\item {} 
\sphinxstyleliteralstrong{calcs} (\sphinxstyleliteralemphasis{dict}) -- dictionary of dictionaries of calculations

\item {} 
\sphinxstyleliteralstrong{variable1} (\sphinxstyleliteralemphasis{str}) -- a string of the variable in the calculations that you want as your x axis

\item {} 
\sphinxstyleliteralstrong{variable2} (\sphinxstyleliteralemphasis{str}) -- a string of the variable in the calculations that you want to your y axis

\end{itemize}

\item[{Keyword Arguments}] \leavevmode\begin{itemize}
\item {} 
\sphinxstyleliteralstrong{title} (\sphinxstyleliteralemphasis{str}) -- Title of plot (default: None)

\item {} 
\sphinxstyleliteralstrong{zaxis} (\sphinxstyleliteralemphasis{str}) -- Choice out of the keywords in each calc to plot in the Z axis (default: Energy)

\item {} 
\sphinxstyleliteralstrong{xaxisTitle} (\sphinxstyleliteralemphasis{str}) -- title of xaxis (default: same as variable1)

\item {} 
\sphinxstyleliteralstrong{yaxisTitle} (\sphinxstyleliteralemphasis{str}) -- title of yaxis (default: same as variable2)

\item {} 
\sphinxstyleliteralstrong{zaxisTitle} (\sphinxstyleliteralemphasis{str}) -- title of zaxis (default: same as zaxis)

\item {} 
\sphinxstyleliteralstrong{fileName} (\sphinxstyleliteralemphasis{str}) -- Name (and path where default is directory where script is run from ) of the
output file (default: `interpolatePlot.pdf')

\item {} 
\sphinxstyleliteralstrong{delta} (\sphinxstyleliteralemphasis{bool}) -- Z-axis scale relative to its first value

\item {} 
\sphinxstyleliteralstrong{delta\_min} (\sphinxstyleliteralemphasis{bool}) -- Z-axis scale relative to its lowest value

\item {} 
\sphinxstyleliteralstrong{text\_min} (\sphinxstyleliteralemphasis{bool}) -- Display text of minimum value next to the minimum value if find\_min=True

\item {} 
\sphinxstyleliteralstrong{vhline\_min} (\sphinxstyleliteralemphasis{bool}) -- Display text of minimum value next to the minimum value if find\_min=True

\item {} 
\sphinxstyleliteralstrong{circle\_min} (\sphinxstyleliteralemphasis{bool}) -- Display text of minimum value next to the minimum value if find\_min=True

\item {} 
\sphinxstyleliteralstrong{delta\_min} -- Display text of minimum value next to the minimum value if find\_min=True

\item {} 
\sphinxstyleliteralstrong{rel\_center} (\sphinxstyleliteralemphasis{bool}) -- Display text of minimum value next to the minimum value if find\_min=True

\item {} 
\sphinxstyleliteralstrong{plot\_color} (\sphinxstyleliteralemphasis{str}) -- string of the matplotlib colormap to be used

\item {} 
\sphinxstyleliteralstrong{find\_min} (\sphinxstyleliteralemphasis{bool}) -- Interpolate between points and find value of
variable1 and variable2 for the minimum value of Z axis

\end{itemize}

\item[{Returns}] \leavevmode
None

\end{description}\end{quote}

\end{fulllineitems}

\index{interpolatePlot1D() (in module plot)}

\begin{fulllineitems}
\phantomsection\label{\detokenize{plot:plot.interpolatePlot1D}}\pysiglinewithargsret{\sphinxcode{plot.}\sphinxbfcode{interpolatePlot1D}}{\emph{calcs}, \emph{variable1}, \emph{yaxis='Energy'}, \emph{xaxisTitle=None}, \emph{yaxisTitle=None}, \emph{title=None}, \emph{fileName='interpolatePlot.pdf'}, \emph{delta=False}, \emph{circle\_min=False}}{}
Takes a list of calculations and plots the energy of the calculations as a function of two input variables
the first value is the baseline for the energy value and the energy plotted is the difference between that
energy and the other energies in the grid
\begin{quote}\begin{description}
\item[{Parameters}] \leavevmode\begin{itemize}
\item {} 
\sphinxstyleliteralstrong{calcs} (\sphinxstyleliteralemphasis{dict}) -- dictionary of dictionaries of calculations

\item {} 
\sphinxstyleliteralstrong{variable1} (\sphinxstyleliteralemphasis{dict}) -- a string of the variable in the calculations that you want as your x axis

\end{itemize}

\item[{Keyword Arguments}] \leavevmode\begin{itemize}
\item {} 
\sphinxstyleliteralstrong{yaxis} (\sphinxstyleliteralemphasis{str}) -- Choice out of the keywords in each calc to plot in the Z axis (default: Energy)

\item {} 
\sphinxstyleliteralstrong{xaxisTitle} (\sphinxstyleliteralemphasis{str}) -- title of xaxis (default: same as variable1)

\item {} 
\sphinxstyleliteralstrong{yaxisTitle} (\sphinxstyleliteralemphasis{str}) -- title of yaxis (default: same as yaxis)

\item {} 
\sphinxstyleliteralstrong{title} (\sphinxstyleliteralemphasis{str}) -- Title of plot (default: None)

\item {} 
\sphinxstyleliteralstrong{fileName} (\sphinxstyleliteralemphasis{str}) -- Name (and path where default is directory where script is run from ) of the
output file (default: `interpolatePlot.pdf')

\item {} 
\sphinxstyleliteralstrong{delta} (\sphinxstyleliteralemphasis{bool}) -- Z-axis scale relative to its first value

\item {} 
\sphinxstyleliteralstrong{circle\_min} (\sphinxstyleliteralemphasis{bool}) -- Display text of minimum value next to the minimum value

\end{itemize}

\end{description}\end{quote}

\end{fulllineitems}

\index{opdos() (in module plot)}

\begin{fulllineitems}
\phantomsection\label{\detokenize{plot:plot.opdos}}\pysiglinewithargsret{\sphinxcode{plot.}\sphinxbfcode{opdos}}{\emph{calcs, yLim={[}-10, 10{]}, runlocal=False, postfix='`, scale=False, tight\_binding=False}}{}
Generates electronic band structure plots for the calculations in the dictionary of dictionaries
of calculations with the option to have a DOS or PDOS plot to accompany it.
\begin{quote}\begin{description}
\item[{Parameters}] \leavevmode
\sphinxstyleliteralstrong{calcs} (\sphinxstyleliteralemphasis{dict}) -- dictionary of dictionaries representing the set of calculations

\item[{Keyword Arguments}] \leavevmode\begin{itemize}
\item {} 
\sphinxstyleliteralstrong{yLim} (\sphinxstyleliteralemphasis{list}) -- List or tuple of two integers for max and min range in horizontal axis of DOS plot

\item {} 
\sphinxstyleliteralstrong{LSDA} (\sphinxstyleliteralemphasis{bool}) -- To plot DOS as spin polarized or not (calculation must have been done as spin polarized)

\item {} 
\sphinxstyleliteralstrong{runlocal} (\sphinxstyleliteralemphasis{bool}) -- Do the plotting right now or if False do it when the calculations are running

\item {} 
\sphinxstyleliteralstrong{postfix} (\sphinxstyleliteralemphasis{str}) -- Postfix to the filename of the plot

\item {} 
\sphinxstyleliteralstrong{tight\_banding} (\sphinxstyleliteralemphasis{bool}) -- Whether to treat the input data as from Quantum Espresso or WanT bands.x

\end{itemize}

\item[{Returns}] \leavevmode
None

\end{description}\end{quote}

\end{fulllineitems}

\index{optical\_plots() (in module plot)}

\begin{fulllineitems}
\phantomsection\label{\detokenize{plot:plot.optical_plots}}\pysiglinewithargsret{\sphinxcode{plot.}\sphinxbfcode{optical\_plots}}{\emph{calcs}, \emph{runlocal=False}, \emph{postfix='`}, \emph{x\_range=None}}{}
\end{fulllineitems}

\index{phonon() (in module plot)}

\begin{fulllineitems}
\phantomsection\label{\detokenize{plot:plot.phonon}}\pysiglinewithargsret{\sphinxcode{plot.}\sphinxbfcode{phonon}}{\emph{calcs}, \emph{runlocal=False}, \emph{postfix='`}, \emph{THz=True}, \emph{color\_accoustic=False}, \emph{color\_optical=False}}{}
\end{fulllineitems}

\index{radialPDF() (in module plot)}

\begin{fulllineitems}
\phantomsection\label{\detokenize{plot:plot.radialPDF}}\pysiglinewithargsret{\sphinxcode{plot.}\sphinxbfcode{radialPDF}}{\emph{calcs, atomNum, filterElement=None, runlocal=False, inpt=False, outp=True, title='`, n\_bins=30, y\_range={[}0, 3{]}, **kwargs}}{}
kwargs get passed to pyplot.hist

\end{fulllineitems}

\index{read\_transport\_datafile() (in module plot)}

\begin{fulllineitems}
\phantomsection\label{\detokenize{plot:plot.read_transport_datafile}}\pysiglinewithargsret{\sphinxcode{plot.}\sphinxbfcode{read\_transport\_datafile}}{\emph{ep\_data\_file}, \emph{mult\_x=1.0}, \emph{mult\_y=1.0}}{}
Arguments:

Keyword Arguments:

Returns:

\end{fulllineitems}

\index{transport\_plots() (in module plot)}

\begin{fulllineitems}
\phantomsection\label{\detokenize{plot:plot.transport_plots}}\pysiglinewithargsret{\sphinxcode{plot.}\sphinxbfcode{transport\_plots}}{\emph{calcs}, \emph{runlocal=False}, \emph{postfix='`}, \emph{x\_range=None}}{}
\end{fulllineitems}



\chapter{prep module}
\label{\detokenize{prep:prep-module}}\label{\detokenize{prep:module-prep}}\label{\detokenize{prep::doc}}\index{prep (module)}\index{ConfigSectionMap() (in module prep)}

\begin{fulllineitems}
\phantomsection\label{\detokenize{prep:prep.ConfigSectionMap}}\pysiglinewithargsret{\sphinxcode{prep.}\sphinxbfcode{ConfigSectionMap}}{\emph{section}, \emph{option}, \emph{configFile=None}}{}
\end{fulllineitems}

\index{addToAll\_() (in module prep)}

\begin{fulllineitems}
\phantomsection\label{\detokenize{prep:prep.addToAll_}}\pysiglinewithargsret{\sphinxcode{prep.}\sphinxbfcode{addToAll\_}}{\emph{calcs}, \emph{block=None}, \emph{addition=None}}{}
\end{fulllineitems}

\index{addToBlockWrapper() (in module prep)}

\begin{fulllineitems}
\phantomsection\label{\detokenize{prep:prep.addToBlockWrapper}}\pysiglinewithargsret{\sphinxcode{prep.}\sphinxbfcode{addToBlockWrapper}}{\emph{oneCalc}, \emph{ID}, \emph{block}, \emph{addition}}{}
Wraps AFLOWpi.prep.\_addToBlock for use inside \_calcs\_container methods
\begin{quote}\begin{description}
\item[{Parameters}] \leavevmode\begin{itemize}
\item {} 
\sphinxstyleliteralstrong{oneCalc} (\sphinxstyleliteralemphasis{dict}) -- a dictionary containing properties about the AFLOWpi calculation

\item {} 
\sphinxstyleliteralstrong{ID} (\sphinxstyleliteralemphasis{str}) -- ID string for the particular calculation and step

\item {} 
\sphinxstyleliteralstrong{block} (\sphinxstyleliteralemphasis{str}) -- string of the block in the \_ID.py that the addition is to be added to
for that step of workflow

\item {} 
\sphinxstyleliteralstrong{addition} (\sphinxstyleliteralemphasis{str}) -- a string containing code to be written to the specific block
in the \_ID.py for each calculation

\end{itemize}

\item[{Returns}] \leavevmode
None

\end{description}\end{quote}

\end{fulllineitems}

\index{askAFLOWpiVars() (in module prep)}

\begin{fulllineitems}
\phantomsection\label{\detokenize{prep:prep.askAFLOWpiVars}}\pysiglinewithargsret{\sphinxcode{prep.}\sphinxbfcode{askAFLOWpiVars}}{\emph{refAFLOWpiVars}}{}
Cycle on the keys of the refAFLOWpiVars dictionary and ask to define them
\begin{quote}\begin{description}
\item[{Parameters}] \leavevmode
\sphinxstyleliteralstrong{refAFLOWpiVars} (\sphinxstyleliteralemphasis{dict}) -- the variables in the ref files that you need to input to run the calculation

\item[{Returns}] \leavevmode
None

\end{description}\end{quote}

\end{fulllineitems}

\index{bands() (in module prep)}

\begin{fulllineitems}
\phantomsection\label{\detokenize{prep:prep.bands}}\pysiglinewithargsret{\sphinxcode{prep.}\sphinxbfcode{bands}}{\emph{calcs}, \emph{dk=None}, \emph{nk=None}, \emph{n\_conduction=None}}{}
Wrapper function to write the function AFLOWpi.prep.\_oneBands to the \_ID.py
\begin{quote}\begin{description}
\item[{Parameters}] \leavevmode
\sphinxstyleliteralstrong{calcs} (\sphinxstyleliteralemphasis{dict}) -- a dictionary of dicionaries representing the set of calculations

\item[{Keyword Arguments}] \leavevmode\begin{itemize}
\item {} 
\sphinxstyleliteralstrong{dk} (\sphinxstyleliteralemphasis{float}) -- distance between points for Electronic Band Structure calculation

\item {} 
\sphinxstyleliteralstrong{nk} (\sphinxstyleliteralemphasis{int}) -- approximate number of k points to be calculated along the path

\end{itemize}

\item[{Returns}] \leavevmode
The identical ``calcs'' input variable

\end{description}\end{quote}

\end{fulllineitems}

\index{bandsAflow() (in module prep)}

\begin{fulllineitems}
\phantomsection\label{\detokenize{prep:prep.bandsAflow}}\pysiglinewithargsret{\sphinxcode{prep.}\sphinxbfcode{bandsAflow}}{\emph{dk}, \emph{LAT}}{}
Query aflow for band structure path and generate the path for band structure calculation
\begin{quote}\begin{description}
\item[{Parameters}] \leavevmode\begin{itemize}
\item {} 
\sphinxstyleliteralstrong{dk} (\sphinxstyleliteralemphasis{float}) -- distance between k points along path in Brillouin Zone

\item {} 
\sphinxstyleliteralstrong{LAT} (\sphinxstyleliteralemphasis{int}) -- bravais lattice number from Quantum Espresso convention

\end{itemize}

\item[{Keyword Arguments}] \leavevmode
\sphinxstyleliteralstrong{None} -- 

\item[{Returns}] \leavevmode
some information
nks (str): number of k points in path
stringk (str): kpoint path string

\item[{Return type}] \leavevmode
info (str)

\end{description}\end{quote}

\end{fulllineitems}

\index{build\_calcs() (in module prep)}

\begin{fulllineitems}
\phantomsection\label{\detokenize{prep:prep.build_calcs}}\pysiglinewithargsret{\sphinxcode{prep.}\sphinxbfcode{build\_calcs}}{\emph{PARAM\_VARS}, \emph{build\_type='product'}}{}
\end{fulllineitems}

\index{calcFromFile() (in module prep)}

\begin{fulllineitems}
\phantomsection\label{\detokenize{prep:prep.calcFromFile}}\pysiglinewithargsret{\sphinxcode{prep.}\sphinxbfcode{calcFromFile}}{\emph{aflowkeys}, \emph{fileList}, \emph{reffile=None}, \emph{pseudodir=None}, \emph{workdir=None}, \emph{keep\_name=False}, \emph{clean\_input=True}, \emph{ref\_override=False}}{}
Reads in a string of an QE input file path, a string of an QE input, a file object of a
QE input or a list of them and attempts to fill create a calculation from them. If they
are missing things such as k\_points card, they are automtically generated.
\begin{quote}\begin{description}
\item[{Parameters}] \leavevmode\begin{itemize}
\item {} 
\sphinxstyleliteralstrong{aflowkeys} (\sphinxstyleliteralemphasis{dict}) -- a dictionary generated by AFLOWpi.prep.init

\item {} 
\sphinxstyleliteralstrong{fileList} (\sphinxstyleliteralemphasis{list}) -- a string of an QE input file path, a string of an QE input, a file
object of a QE input or a list of them

\end{itemize}

\item[{Keyword Arguments}] \leavevmode\begin{itemize}
\item {} 
\sphinxstyleliteralstrong{reffile} (\sphinxstyleliteralemphasis{str}) -- a partially filled QE input file used in case the input(s) in fileList
are missing. i.e. wfc cutoff. If the names of the Pseudopotential files
are not included in the input(s) in fileList, they are chosen depending
on the pseudodir chosen and included when the calculation set is formed.

\item {} 
\sphinxstyleliteralstrong{workdir} (\sphinxstyleliteralemphasis{str}) -- a string of the workdir path that be used to override what is in the
config file used when initating the AFLOWpi session

\item {} 
\sphinxstyleliteralstrong{pseudodir} (\sphinxstyleliteralemphasis{str}) -- a string of the pseudodir path that be used to override what is in
the config file used when initating the AFLOWpi session

\end{itemize}

\end{description}\end{quote}

\end{fulllineitems}

\index{calcs\_container (class in prep)}

\begin{fulllineitems}
\phantomsection\label{\detokenize{prep:prep.calcs_container}}\pysiglinewithargsret{\sphinxstrong{class }\sphinxcode{prep.}\sphinxbfcode{calcs\_container}}{\emph{dictionary}}{}~\index{acbn0() (prep.calcs\_container method)}

\begin{fulllineitems}
\phantomsection\label{\detokenize{prep:prep.calcs_container.acbn0}}\pysiglinewithargsret{\sphinxbfcode{acbn0}}{\emph{thresh=0.1}, \emph{nIters=20}, \emph{paodir=None}, \emph{relax='scf'}, \emph{mixing=0.2}, \emph{kp\_mult=1.5}}{}
Wrapper method to call AFLOWpi.scfuj.scfPrep and AFLOWpi.scfuj.run in the high level
user interface. Adds a new step to the workflow.
\begin{quote}\begin{description}
\item[{Parameters}] \leavevmode
\sphinxstyleliteralstrong{self} -- the \_calcs\_container object

\item[{Keyword Arguments}] \leavevmode\begin{itemize}
\item {} 
\sphinxstyleliteralstrong{thresh} (\sphinxstyleliteralemphasis{float}) -- threshold for self consistent hubbard U convergence

\item {} 
\sphinxstyleliteralstrong{niters} (\sphinxstyleliteralemphasis{int}) -- max number of iterations of the acbn0 cycle

\item {} 
\sphinxstyleliteralstrong{paodir} (\sphinxstyleliteralemphasis{string}) -- the path of the PAO directory. This will override
an entry of the paodir in the AFLOWpi config file
used for the session

\item {} 
\sphinxstyleliteralstrong{mixing} (\sphinxstyleliteralemphasis{float}) -- the amount of the previous acbn0 U iteration to mix into
the current (only needed when there is U val oscillation)

\end{itemize}

\item[{Returns}] \leavevmode
None

\end{description}\end{quote}

\end{fulllineitems}

\index{addToAll() (prep.calcs\_container method)}

\begin{fulllineitems}
\phantomsection\label{\detokenize{prep:prep.calcs_container.addToAll}}\pysiglinewithargsret{\sphinxbfcode{addToAll}}{\emph{block=None}, \emph{addition=None}}{}
\end{fulllineitems}

\index{bands() (prep.calcs\_container method)}

\begin{fulllineitems}
\phantomsection\label{\detokenize{prep:prep.calcs_container.bands}}\pysiglinewithargsret{\sphinxbfcode{bands}}{\emph{dk=None}, \emph{nk=100}, \emph{n\_conduction=None}}{}
Wrapper method to write call AFLOWpi.prep.bands for calculating the Electronic Band
Structure.
\begin{quote}\begin{description}
\item[{Parameters}] \leavevmode
\sphinxstyleliteralstrong{calcs} (\sphinxstyleliteralemphasis{dict}) -- a dictionary of dicionaries representing the set of calculations

\item[{Keyword Arguments}] \leavevmode\begin{itemize}
\item {} 
\sphinxstyleliteralstrong{dk} (\sphinxstyleliteralemphasis{float}) -- the density in the Brillouin zone of the k point sampling along the
entirety of the path between high symmetry points.

\item {} 
\sphinxstyleliteralstrong{nk} (\sphinxstyleliteralemphasis{int}) -- the approximate number of sampling points  in the Brillouin Zone along
the entirety of the path between high symmetry points. Points are
chosen so that they are equidistant along the entirety of the path.
The actual number of points will be slightly different than the
inputted value of nk. nk!=None will override any value for dk.

\end{itemize}

\item[{Returns}] \leavevmode
None

\end{description}\end{quote}

\end{fulllineitems}

\index{change\_input() (prep.calcs\_container method)}

\begin{fulllineitems}
\phantomsection\label{\detokenize{prep:prep.calcs_container.change_input}}\pysiglinewithargsret{\sphinxbfcode{change\_input}}{\emph{namelist=None}, \emph{parameter=None}, \emph{value=None}}{}
\end{fulllineitems}

\index{change\_pseudos() (prep.calcs\_container method)}

\begin{fulllineitems}
\phantomsection\label{\detokenize{prep:prep.calcs_container.change_pseudos}}\pysiglinewithargsret{\sphinxbfcode{change\_pseudos}}{\emph{directory}}{}
\end{fulllineitems}

\index{conventional\_cell\_input() (prep.calcs\_container method)}

\begin{fulllineitems}
\phantomsection\label{\detokenize{prep:prep.calcs_container.conventional_cell_input}}\pysiglinewithargsret{\sphinxbfcode{conventional\_cell\_input}}{}{}
\end{fulllineitems}

\index{converge\_smearing() (prep.calcs\_container method)}

\begin{fulllineitems}
\phantomsection\label{\detokenize{prep:prep.calcs_container.converge_smearing}}\pysiglinewithargsret{\sphinxbfcode{converge\_smearing}}{\emph{relax='scf'}, \emph{smear\_variance=0.3}, \emph{num\_points=4}, \emph{smear\_type='mp'}, \emph{mult\_jobs=False}}{}
\end{fulllineitems}

\index{crawl\_min() (prep.calcs\_container method)}

\begin{fulllineitems}
\phantomsection\label{\detokenize{prep:prep.calcs_container.crawl_min}}\pysiglinewithargsret{\sphinxbfcode{crawl\_min}}{\emph{mult\_jobs=False}, \emph{grid\_density=10}, \emph{initial\_variance=0.02}, \emph{thresh=0.01}, \emph{constraint=None}, \emph{final\_minimization='relax'}}{}
Wrapper method to call AFLOWpi.pseudo.crawlingMinimization in the high level user interface.
Adds a new step to the workflow.
\begin{quote}\begin{description}
\item[{Parameters}] \leavevmode
\sphinxstyleliteralstrong{self} -- the \_calcs\_container object

\item[{Keyword Arguments}] \leavevmode\begin{itemize}
\item {} 
\sphinxstyleliteralstrong{mult\_jobs} (\sphinxstyleliteralemphasis{bool}) -- if True split the individual scf jobs into separate cluster jobs
if False run them serially

\item {} 
\sphinxstyleliteralstrong{grid\_density} (\sphinxstyleliteralemphasis{int}) -- controls the number of calculations to generate for the minimization
num\_\{calcs\}=grid\_density\textasciicircum{}\{num\_\{parameters\}-num\_\{constraints\}\}

\item {} 
\sphinxstyleliteralstrong{initial\_variance} (\sphinxstyleliteralemphasis{float}) -- amount to vary the values of the parameters from the initial
value. i.e. (0.02 = +/-2\% variance)

\item {} 
\sphinxstyleliteralstrong{thresh} (\sphinxstyleliteralemphasis{float}) -- threshold for \$DeltaX\$ of the lattice parameters between brute force
minimization iterations.

\item {} 
\sphinxstyleliteralstrong{constraint} (\sphinxstyleliteralemphasis{list}) -- a list or tuple containing two entry long list or tuples with
the first being the constraint type and the second the free
parameter in params that its constraining for example in a
orthorhombic cell: constraint=({[}''volume'','c'{]},) allows for A and B
to move freely but C is such that it keeps the cell volume the same
in all calculations generated by the input oneCalc calculation.

\item {} 
\sphinxstyleliteralstrong{final\_minimization} (\sphinxstyleliteralemphasis{str}) -- calculation to be run at the end of the brute force minimization
options include ``scf'', ``relax'', and ``vcrelax''

\end{itemize}

\item[{Returns}] \leavevmode
None

\end{description}\end{quote}

\end{fulllineitems}

\index{dos() (prep.calcs\_container method)}

\begin{fulllineitems}
\phantomsection\label{\detokenize{prep:prep.calcs_container.dos}}\pysiglinewithargsret{\sphinxbfcode{dos}}{\emph{kpFactor=2}, \emph{project=True}, \emph{n\_conduction=None}}{}
Wrapper method to call AFLOWpi.prep.doss in the high level user interface.
Adds a new step to the workflow.
\begin{quote}\begin{description}
\item[{Parameters}] \leavevmode
\sphinxstyleliteralstrong{self} -- the \_calcs\_container object

\item[{Keyword Arguments}] \leavevmode\begin{itemize}
\item {} 
\sphinxstyleliteralstrong{kpFactor} (\sphinxstyleliteralemphasis{float}) -- factor to which the k-point grid is made denser in each direction

\item {} 
\sphinxstyleliteralstrong{project} (\sphinxstyleliteralemphasis{bool}) -- if True: do the projected DOS after completing the DOS

\end{itemize}

\item[{Returns}] \leavevmode
None

\end{description}\end{quote}

\end{fulllineitems}

\index{elastic() (prep.calcs\_container method)}

\begin{fulllineitems}
\phantomsection\label{\detokenize{prep:prep.calcs_container.elastic}}\pysiglinewithargsret{\sphinxbfcode{elastic}}{\emph{mult\_jobs=False}, \emph{order=2}, \emph{eta\_max=0.005}, \emph{num\_dist=10}}{}
\end{fulllineitems}

\index{evCurve\_min() (prep.calcs\_container method)}

\begin{fulllineitems}
\phantomsection\label{\detokenize{prep:prep.calcs_container.evCurve_min}}\pysiglinewithargsret{\sphinxbfcode{evCurve\_min}}{\emph{pThresh=25}, \emph{final\_minimization='relax'}}{}
\end{fulllineitems}

\index{get\_initial\_inputs() (prep.calcs\_container method)}

\begin{fulllineitems}
\phantomsection\label{\detokenize{prep:prep.calcs_container.get_initial_inputs}}\pysiglinewithargsret{\sphinxbfcode{get\_initial\_inputs}}{}{}
\end{fulllineitems}

\index{increase\_step() (prep.calcs\_container method)}

\begin{fulllineitems}
\phantomsection\label{\detokenize{prep:prep.calcs_container.increase_step}}\pysiglinewithargsret{\sphinxbfcode{increase\_step}}{\emph{func}}{}
\end{fulllineitems}

\index{items() (prep.calcs\_container method)}

\begin{fulllineitems}
\phantomsection\label{\detokenize{prep:prep.calcs_container.items}}\pysiglinewithargsret{\sphinxbfcode{items}}{}{}
\end{fulllineitems}

\index{iteritems() (prep.calcs\_container method)}

\begin{fulllineitems}
\phantomsection\label{\detokenize{prep:prep.calcs_container.iteritems}}\pysiglinewithargsret{\sphinxbfcode{iteritems}}{}{}
\end{fulllineitems}

\index{keys() (prep.calcs\_container method)}

\begin{fulllineitems}
\phantomsection\label{\detokenize{prep:prep.calcs_container.keys}}\pysiglinewithargsret{\sphinxbfcode{keys}}{}{}
\end{fulllineitems}

\index{new\_step() (prep.calcs\_container method)}

\begin{fulllineitems}
\phantomsection\label{\detokenize{prep:prep.calcs_container.new_step}}\pysiglinewithargsret{\sphinxbfcode{new\_step}}{\emph{update\_positions=True}, \emph{update\_structure=True}, \emph{new\_job=True}, \emph{ext='`}}{}
\end{fulllineitems}

\index{phonon() (prep.calcs\_container method)}

\begin{fulllineitems}
\phantomsection\label{\detokenize{prep:prep.calcs_container.phonon}}\pysiglinewithargsret{\sphinxbfcode{phonon}}{\emph{nrx1=2}, \emph{nrx2=2}, \emph{nrx3=2}, \emph{innx=2}, \emph{de=0.003}, \emph{mult\_jobs=False}, \emph{LOTO=True}, \emph{disp\_sym=False}, \emph{atom\_sym=False}, \emph{field\_strength=0.003}, \emph{field\_cycles=3}, \emph{proj\_phDOS=True}}{}
\end{fulllineitems}

\index{pseudo\_test\_brute() (prep.calcs\_container method)}

\begin{fulllineitems}
\phantomsection\label{\detokenize{prep:prep.calcs_container.pseudo_test_brute}}\pysiglinewithargsret{\sphinxbfcode{pseudo\_test\_brute}}{\emph{ecutwfc}, \emph{dual={[}{]}}, \emph{sampling={[}{]}}, \emph{conv\_thresh=0.01}, \emph{constraint=None}, \emph{initial\_relax=None}, \emph{min\_thresh=0.01}, \emph{initial\_variance=0.05}, \emph{grid\_density=7}, \emph{mult\_jobs=False}, \emph{options=None}}{}
\end{fulllineitems}

\index{relax() (prep.calcs\_container method)}

\begin{fulllineitems}
\phantomsection\label{\detokenize{prep:prep.calcs_container.relax}}\pysiglinewithargsret{\sphinxbfcode{relax}}{}{}
\end{fulllineitems}

\index{resubmit() (prep.calcs\_container method)}

\begin{fulllineitems}
\phantomsection\label{\detokenize{prep:prep.calcs_container.resubmit}}\pysiglinewithargsret{\sphinxbfcode{resubmit}}{\emph{reset=True}}{}
\end{fulllineitems}

\index{scf() (prep.calcs\_container method)}

\begin{fulllineitems}
\phantomsection\label{\detokenize{prep:prep.calcs_container.scf}}\pysiglinewithargsret{\sphinxbfcode{scf}}{}{}
\end{fulllineitems}

\index{submit() (prep.calcs\_container method)}

\begin{fulllineitems}
\phantomsection\label{\detokenize{prep:prep.calcs_container.submit}}\pysiglinewithargsret{\sphinxbfcode{submit}}{}{}
\end{fulllineitems}

\index{thermal() (prep.calcs\_container method)}

\begin{fulllineitems}
\phantomsection\label{\detokenize{prep:prep.calcs_container.thermal}}\pysiglinewithargsret{\sphinxbfcode{thermal}}{\emph{delta\_volume=0.03}, \emph{nrx1=2}, \emph{nrx2=2}, \emph{nrx3=2}, \emph{innx=2}, \emph{de=0.005}, \emph{mult\_jobs=False}, \emph{disp\_sym=False}, \emph{atom\_sym=False}, \emph{field\_strength=0.001}, \emph{field\_cycles=3}, \emph{LOTO=True}, \emph{hydrostatic\_expansion=True}}{}
\end{fulllineitems}

\index{tight\_binding() (prep.calcs\_container method)}

\begin{fulllineitems}
\phantomsection\label{\detokenize{prep:prep.calcs_container.tight_binding}}\pysiglinewithargsret{\sphinxbfcode{tight\_binding}}{\emph{cond\_bands=True}, \emph{proj\_thr=0.95}, \emph{kp\_factor=2.0}, \emph{proj\_sh=5.5}, \emph{cond\_bands\_proj=True}}{}
\end{fulllineitems}

\index{values() (prep.calcs\_container method)}

\begin{fulllineitems}
\phantomsection\label{\detokenize{prep:prep.calcs_container.values}}\pysiglinewithargsret{\sphinxbfcode{values}}{}{}
\end{fulllineitems}

\index{vcrelax() (prep.calcs\_container method)}

\begin{fulllineitems}
\phantomsection\label{\detokenize{prep:prep.calcs_container.vcrelax}}\pysiglinewithargsret{\sphinxbfcode{vcrelax}}{}{}
\end{fulllineitems}


\end{fulllineitems}

\index{changeCalcs() (in module prep)}

\begin{fulllineitems}
\phantomsection\label{\detokenize{prep:prep.changeCalcs}}\pysiglinewithargsret{\sphinxcode{prep.}\sphinxbfcode{changeCalcs}}{\emph{calcs}, \emph{keyword='calculation'}, \emph{value='scf'}}{}
A Wrapper function that writes the function AFLOWpi.prep.\_changeCalcs to the \_ID.py
\begin{quote}\begin{description}
\item[{Parameters}] \leavevmode
\sphinxstyleliteralstrong{calcs} (\sphinxstyleliteralemphasis{dict}) -- a dictionary of dicionaries representing the set of calculations

\item[{Keyword Arguments}] \leavevmode\begin{itemize}
\item {} 
\sphinxstyleliteralstrong{keyword} (\sphinxstyleliteralemphasis{str}) -- a string which signifies the type of change that is to be made

\item {} 
\sphinxstyleliteralstrong{value} -- the value of the choice.

\end{itemize}

\item[{Returns}] \leavevmode
The identical set of calculations as the input to this function

\end{description}\end{quote}

\end{fulllineitems}

\index{cleanCalcs() (in module prep)}

\begin{fulllineitems}
\phantomsection\label{\detokenize{prep:prep.cleanCalcs}}\pysiglinewithargsret{\sphinxcode{prep.}\sphinxbfcode{cleanCalcs}}{\emph{calcs}, \emph{runlocal=False}}{}
Wrapper function for AFLOWpi.prep.\_cleanCalcs
\begin{quote}\begin{description}
\item[{Parameters}] \leavevmode
\sphinxstyleliteralstrong{calcs} (\sphinxstyleliteralemphasis{dict}) -- a dictionary of dicionaries representing the set of calculations

\item[{Keyword Arguments}] \leavevmode
\sphinxstyleliteralstrong{runlocal} (\sphinxstyleliteralemphasis{bool}) -- a flag to choose whether or not to run the wrapped function now
or write it to the \_ID.py to run during the workflow

\item[{Returns}] \leavevmode
None

\end{description}\end{quote}

\end{fulllineitems}

\index{construct\_and\_run() (in module prep)}

\begin{fulllineitems}
\phantomsection\label{\detokenize{prep:prep.construct_and_run}}\pysiglinewithargsret{\sphinxcode{prep.}\sphinxbfcode{construct\_and\_run}}{\emph{\_\_submitNodeName\_\_}, \emph{oneCalc}, \emph{ID}, \emph{build\_command='`}, \emph{subset\_tasks={[}{]}}, \emph{fault\_tolerant=False}, \emph{mult\_jobs=True}, \emph{subset\_name='SUBSET'}, \emph{keep\_file\_names=False}, \emph{clean\_input=True}, \emph{check\_function=None}}{}
\end{fulllineitems}

\index{doss() (in module prep)}

\begin{fulllineitems}
\phantomsection\label{\detokenize{prep:prep.doss}}\pysiglinewithargsret{\sphinxcode{prep.}\sphinxbfcode{doss}}{\emph{calcs}, \emph{kpFactor=1.5}, \emph{n\_conduction=None}}{}
Wrapper function to write the functio n AFLOWpi.prep.\_oneDoss to the \_ID.py
\begin{quote}\begin{description}
\item[{Parameters}] \leavevmode
\sphinxstyleliteralstrong{calcs} (\sphinxstyleliteralemphasis{dict}) -- a dictionary of dicionaries representing the set of calculations

\item[{Keyword Arguments}] \leavevmode
\sphinxstyleliteralstrong{kpFactor} (\sphinxstyleliteralemphasis{float}) -- the factor to which we make each direction in the kpoint grid denser

\item[{Returns}] \leavevmode
The identical ``calcs'' input variable

\end{description}\end{quote}

\end{fulllineitems}

\index{extractvars() (in module prep)}

\begin{fulllineitems}
\phantomsection\label{\detokenize{prep:prep.extractvars}}\pysiglinewithargsret{\sphinxcode{prep.}\sphinxbfcode{extractvars}}{\emph{refFile}}{}
Read refFile and return an empty dictionary with all the keys and None values
\begin{quote}\begin{description}
\item[{Parameters}] \leavevmode
\sphinxstyleliteralstrong{refFile} (\sphinxstyleliteralemphasis{str}) -- filename of the reference input file for the frame

\item[{Returns}] \leavevmode
A dictionary containing the keyword extracted from the reference input file
as keys with None for values..

\end{description}\end{quote}

\end{fulllineitems}

\index{generateAnotherCalc() (in module prep)}

\begin{fulllineitems}
\phantomsection\label{\detokenize{prep:prep.generateAnotherCalc}}\pysiglinewithargsret{\sphinxcode{prep.}\sphinxbfcode{generateAnotherCalc}}{\emph{old}, \emph{new}, \emph{calcs}}{}
Modify the calculation in each subdir and update the master dictionary
\begin{quote}\begin{description}
\item[{Parameters}] \leavevmode\begin{itemize}
\item {} 
\sphinxstyleliteralstrong{old} (\sphinxstyleliteralemphasis{str}) -- string to replace

\item {} 
\sphinxstyleliteralstrong{new} (\sphinxstyleliteralemphasis{str}) -- replacement string

\item {} 
\sphinxstyleliteralstrong{calcs} (\sphinxstyleliteralemphasis{dict}) -- dictionary of dictionaries of calculations

\end{itemize}

\item[{Returns}] \leavevmode
A new set of calculations with a new ID of the hash of the new input strings

\end{description}\end{quote}

\end{fulllineitems}

\index{getMPGrid() (in module prep)}

\begin{fulllineitems}
\phantomsection\label{\detokenize{prep:prep.getMPGrid}}\pysiglinewithargsret{\sphinxcode{prep.}\sphinxbfcode{getMPGrid}}{\emph{primLatVec}, \emph{offset=True}, \emph{string=True}}{}
\end{fulllineitems}

\index{init (class in prep)}

\begin{fulllineitems}
\phantomsection\label{\detokenize{prep:prep.init}}\pysiglinewithargsret{\sphinxstrong{class }\sphinxcode{prep.}\sphinxbfcode{init}}{\emph{PROJECT}, \emph{SET='`}, \emph{AUTHOR='`}, \emph{CORRESPONDING='`}, \emph{SPONSOR='`}, \emph{config='`}, \emph{workdir=None}, \emph{make\_symlink=False}}{}~\index{from\_file() (prep.init method)}

\begin{fulllineitems}
\phantomsection\label{\detokenize{prep:prep.init.from_file}}\pysiglinewithargsret{\sphinxbfcode{from\_file}}{\emph{fileList}, \emph{reffile=None}, \emph{pseudodir=None}, \emph{workdir=None}, \emph{ref\_override=True}}{}
Reads in a string of an QE input file path, a string of an QE input, a file object of a
QE input or a list of them and attempts to fill create a calculation from them. If they
are missing things such as k\_points card, they are automtically generated.
\begin{quote}\begin{description}
\item[{Parameters}] \leavevmode\begin{itemize}
\item {} 
\sphinxstyleliteralstrong{aflowkeys} (\sphinxstyleliteralemphasis{dict}) -- a dictionary generated by AFLOWpi.prep.init

\item {} 
\sphinxstyleliteralstrong{fileList} (\sphinxstyleliteralemphasis{str}) -- a string of an QE input file path, a string of an QE input, a file
object of a QE input or a list of them

\end{itemize}

\item[{Keyword Arguments}] \leavevmode\begin{itemize}
\item {} 
\sphinxstyleliteralstrong{reffile} (\sphinxstyleliteralemphasis{str}) -- a partially filled QE input file used in case the input(s) in fileList
are missing. i.e. wfc cutoff. If the names of the Pseudopotential files
are not included in the input(s) in fileList, they are chosen depending
on the pseudodir chosen and included when the calculation set is formed.

\item {} 
\sphinxstyleliteralstrong{workdir} (\sphinxstyleliteralemphasis{str}) -- a string of the workdir path that be used to override what is in the
config file used when initating the AFLOWpi session

\item {} 
\sphinxstyleliteralstrong{pseudodir} (\sphinxstyleliteralemphasis{str}) -- a string of the pseudodir path that be used to override what is in
the config file used when initating the AFLOWpi session

\item {} 
\sphinxstyleliteralstrong{ref\_override} (\sphinxstyleliteralemphasis{bool}) -- Option to override values in the input file(s) with values in the reference
input file. If no reference input file is included then this is ignored.
(Default: True)

\end{itemize}

\end{description}\end{quote}

\end{fulllineitems}

\index{items() (prep.init method)}

\begin{fulllineitems}
\phantomsection\label{\detokenize{prep:prep.init.items}}\pysiglinewithargsret{\sphinxbfcode{items}}{}{}
\end{fulllineitems}

\index{iteritems() (prep.init method)}

\begin{fulllineitems}
\phantomsection\label{\detokenize{prep:prep.init.iteritems}}\pysiglinewithargsret{\sphinxbfcode{iteritems}}{}{}
\end{fulllineitems}

\index{keys() (prep.init method)}

\begin{fulllineitems}
\phantomsection\label{\detokenize{prep:prep.init.keys}}\pysiglinewithargsret{\sphinxbfcode{keys}}{}{}
\end{fulllineitems}

\index{load() (prep.init method)}

\begin{fulllineitems}
\phantomsection\label{\detokenize{prep:prep.init.load}}\pysiglinewithargsret{\sphinxbfcode{load}}{\emph{step=1}}{}
Loads the calc logs from a given step
\begin{quote}\begin{description}
\item[{Parameters}] \leavevmode
\sphinxstyleliteralstrong{step} (\sphinxstyleliteralemphasis{int}) -- the step of the calculation for whose calclogs are to be loaded

\item[{Returns}] \leavevmode
the loaded calc logs

\item[{Return type}] \leavevmode
calcs (dict)

\end{description}\end{quote}

\end{fulllineitems}

\index{scfs() (prep.init method)}

\begin{fulllineitems}
\phantomsection\label{\detokenize{prep:prep.init.scfs}}\pysiglinewithargsret{\sphinxbfcode{scfs}}{\emph{allAFLOWpiVars}, \emph{refFile}, \emph{name='first'}, \emph{pseudodir=None}, \emph{build\_type='product'}, \emph{convert=True}}{}
A wrapper method to call AFLOWpi.prep.scfs to form the calculation set. This will
also create directory within the set directory for every calculation in the set.
\begin{quote}\begin{description}
\item[{Parameters}] \leavevmode\begin{itemize}
\item {} 
\sphinxstyleliteralstrong{allAFLOWpiVars} (\sphinxstyleliteralemphasis{dict}) -- a dictionary whose keys correspond to the keywords in the
reference input file and whose values will be used to
construct the set of calculations

\item {} 
\sphinxstyleliteralstrong{refFile} (\sphinxstyleliteralemphasis{str}) -- a filename as a string, a file object, or a string of the file
that contains keywords to construct the inputs to the different
calculations in the set

\end{itemize}

\item[{Keyword Arguments}] \leavevmode\begin{itemize}
\item {} 
\sphinxstyleliteralstrong{pseudodir} (\sphinxstyleliteralemphasis{str}) -- path of the directory that contains your Pseudopotential files
The value in the AFLOWpi config file used will override this.

\item {} 
\sphinxstyleliteralstrong{build\_type} (\sphinxstyleliteralemphasis{str}) -- 
how to construct the calculation set from allAFLOWpiVars dictionary:
\begin{description}
\item[{zip \textbar{} The first calculation takes the first entry from the list of}] \leavevmode
\begin{DUlineblock}{0em}
\item[] each of the keywords. The second calculation takes the second
\item[] and so on. The keywords for all lists in allAFLOWpiVars must be
\item[] the same length for this method.
\end{DUlineblock}

\item[{product \textbar{} Calculation set is formed via a ``cartesian product'' with}] \leavevmode
\begin{DUlineblock}{0em}
\item[] the values the list of each keyword combined. (i.e if
\item[] allAFLOWpiVars has one keyword with a list of 5 entires and
\item[] another with 4 and a third with 10, there would be 2000
\item[] calculations in the set formed from them via product mode.
\end{DUlineblock}

\end{description}


\end{itemize}

\item[{Returns}] \leavevmode
A dictionary of dictionaries containing the set of calculations.

\end{description}\end{quote}

\end{fulllineitems}

\index{status() (prep.init method)}

\begin{fulllineitems}
\phantomsection\label{\detokenize{prep:prep.init.status}}\pysiglinewithargsret{\sphinxbfcode{status}}{\emph{status=\{\}}, \emph{step=0}, \emph{negate\_status=False}}{}
Loads the calc logs from a given step
\begin{quote}\begin{description}
\item[{Parameters}] \leavevmode
\sphinxstyleliteralstrong{step} (\sphinxstyleliteralemphasis{int}) -- The step of the calculation for whose calclogs are to be loaded.
If no step is specified then it will default to load calculations
from all steps with the chosen status.

\item[{Keyword Arguments}] \leavevmode\begin{itemize}
\item {} 
\sphinxstyleliteralstrong{status} (\sphinxstyleliteralemphasis{dict}) -- key,value pairs for status type and their value to filter on.
i.e. status=\{`Finished':False\}

\item {} 
\sphinxstyleliteralstrong{negate\_status} (\sphinxstyleliteralemphasis{bool}) -- filter on the opposite of the status filters

\end{itemize}

\item[{Returns}] \leavevmode
the loaded calcs for one or more steps with the given status

\item[{Return type}] \leavevmode
calcs (dict)

\end{description}\end{quote}

\end{fulllineitems}

\index{values() (prep.init method)}

\begin{fulllineitems}
\phantomsection\label{\detokenize{prep:prep.init.values}}\pysiglinewithargsret{\sphinxbfcode{values}}{}{}
\end{fulllineitems}


\end{fulllineitems}

\index{init\_\_() (in module prep)}

\begin{fulllineitems}
\phantomsection\label{\detokenize{prep:prep.init__}}\pysiglinewithargsret{\sphinxcode{prep.}\sphinxbfcode{init\_\_}}{\emph{PROJECT}, \emph{SET='`}, \emph{AUTHOR='`}, \emph{CORRESPONDING='`}, \emph{SPONSOR='`}, \emph{config='`}, \emph{workdir=None}, \emph{make\_symlink=False}}{}
Initializes the frame
\begin{quote}\begin{description}
\item[{Parameters}] \leavevmode\begin{itemize}
\item {} 
\sphinxstyleliteralstrong{PROJECT} (\sphinxstyleliteralemphasis{str}) -- Name of project

\item {} 
\sphinxstyleliteralstrong{SET} (\sphinxstyleliteralemphasis{str}) -- Name of set

\item {} 
\sphinxstyleliteralstrong{author} (\sphinxstyleliteralemphasis{str}) -- Name of author

\item {} 
\sphinxstyleliteralstrong{CORRESPONDING} (\sphinxstyleliteralemphasis{str}) -- Name of corresponding

\item {} 
\sphinxstyleliteralstrong{SPONSOR} (\sphinxstyleliteralemphasis{str}) -- Name of sponsor

\end{itemize}

\end{description}\end{quote}

e.g. initFrame(`LNTYPE','`, `MF', \href{mailto:'marco.fornari@cmich.edu}{`marco.fornari@cmich.edu}`,'DOD-MURI'). Return the AFLOKEYS dictionary.

\end{fulllineitems}

\index{isotropy (class in prep)}

\begin{fulllineitems}
\phantomsection\label{\detokenize{prep:prep.isotropy}}\pysiglinewithargsret{\sphinxstrong{class }\sphinxcode{prep.}\sphinxbfcode{isotropy}}{\emph{input\_str}, \emph{accuracy=0.001}, \emph{output=False}}{}~\index{cif2qe() (prep.isotropy method)}

\begin{fulllineitems}
\phantomsection\label{\detokenize{prep:prep.isotropy.cif2qe}}\pysiglinewithargsret{\sphinxbfcode{cif2qe}}{}{}
\end{fulllineitems}

\index{convert() (prep.isotropy method)}

\begin{fulllineitems}
\phantomsection\label{\detokenize{prep:prep.isotropy.convert}}\pysiglinewithargsret{\sphinxbfcode{convert}}{\emph{ibrav=True}}{}
\end{fulllineitems}

\index{get\_cif() (prep.isotropy method)}

\begin{fulllineitems}
\phantomsection\label{\detokenize{prep:prep.isotropy.get_cif}}\pysiglinewithargsret{\sphinxbfcode{get\_cif}}{}{}
\end{fulllineitems}


\end{fulllineitems}

\index{line\_prepender() (in module prep)}

\begin{fulllineitems}
\phantomsection\label{\detokenize{prep:prep.line_prepender}}\pysiglinewithargsret{\sphinxcode{prep.}\sphinxbfcode{line\_prepender}}{\emph{filename}, \emph{new\_text}}{}
prepends a file with a new line containing the contents of the string new\_text.
\begin{quote}\begin{description}
\item[{Parameters}] \leavevmode\begin{itemize}
\item {} 
\sphinxstyleliteralstrong{filename} (\sphinxstyleliteralemphasis{str}) -- string of the filename that is to be prepended

\item {} 
\sphinxstyleliteralstrong{new\_text} (\sphinxstyleliteralemphasis{str}) -- a string that is one line long to be prepended to the file

\end{itemize}

\item[{Returns}] \leavevmode
None

\end{description}\end{quote}

\end{fulllineitems}

\index{loadlogs() (in module prep)}

\begin{fulllineitems}
\phantomsection\label{\detokenize{prep:prep.loadlogs}}\pysiglinewithargsret{\sphinxcode{prep.}\sphinxbfcode{loadlogs}}{\emph{PROJECT='`}, \emph{SET='`}, \emph{logname='`}, \emph{config=None}, \emph{suppress\_warning=False}}{}
\end{fulllineitems}

\index{lockAtomMovement() (in module prep)}

\begin{fulllineitems}
\phantomsection\label{\detokenize{prep:prep.lockAtomMovement}}\pysiglinewithargsret{\sphinxcode{prep.}\sphinxbfcode{lockAtomMovement}}{\emph{calcs}}{}
A Wrapper function that writes the function AFLOWpi.prep.\_freezeAtoms to the \_ID.py
\begin{quote}\begin{description}
\item[{Parameters}] \leavevmode
\sphinxstyleliteralstrong{calcs} (\sphinxstyleliteralemphasis{dict}) -- a dictionary of dicionaries representing the set of calculations

\item[{Returns}] \leavevmode
None

\end{description}\end{quote}

\end{fulllineitems}

\index{maketree() (in module prep)}

\begin{fulllineitems}
\phantomsection\label{\detokenize{prep:prep.maketree}}\pysiglinewithargsret{\sphinxcode{prep.}\sphinxbfcode{maketree}}{\emph{calcs}, \emph{pseudodir=None}, \emph{workdir=None}}{}
Make the directoy tree and place in the input file there
\begin{quote}\begin{description}
\item[{Parameters}] \leavevmode
\sphinxstyleliteralstrong{calcs} (\sphinxstyleliteralemphasis{dict}) -- \begin{itemize}
\item {} 
Dictionary of dictionaries of calculations

\end{itemize}


\item[{Keyword Arguments}] \leavevmode\begin{itemize}
\item {} 
\sphinxstyleliteralstrong{pseudodir} (\sphinxstyleliteralemphasis{str}) -- path of pseudopotential files directory

\item {} 
\sphinxstyleliteralstrong{workdir} (\sphinxstyleliteralemphasis{str}) -- a string of the workdir path that be used to override what is in the
config file used when initating the AFLOWpi session

\end{itemize}

\item[{Returns}] \leavevmode
None

\end{description}\end{quote}

\end{fulllineitems}

\index{modifyCalcs() (in module prep)}

\begin{fulllineitems}
\phantomsection\label{\detokenize{prep:prep.modifyCalcs}}\pysiglinewithargsret{\sphinxcode{prep.}\sphinxbfcode{modifyCalcs}}{\emph{old}, \emph{new}, \emph{calcs}}{}
Modify the calculation in each subdir and update the master dictionary
\begin{quote}\begin{description}
\item[{Parameters}] \leavevmode\begin{itemize}
\item {} 
\sphinxstyleliteralstrong{old} (\sphinxstyleliteralemphasis{str}) -- string to replace

\item {} 
\sphinxstyleliteralstrong{new} (\sphinxstyleliteralemphasis{str}) -- replacement string

\item {} 
\sphinxstyleliteralstrong{calcs} (\sphinxstyleliteralemphasis{dict}) -- dictionary of dictionaries of calculations

\end{itemize}

\end{description}\end{quote}

\end{fulllineitems}

\index{modifyInputPrefixPW() (in module prep)}

\begin{fulllineitems}
\phantomsection\label{\detokenize{prep:prep.modifyInputPrefixPW}}\pysiglinewithargsret{\sphinxcode{prep.}\sphinxbfcode{modifyInputPrefixPW}}{\emph{calcs}, \emph{pre}}{}
A Wrapper function that is used to write a function to the \_ID.py
\begin{quote}\begin{description}
\item[{Parameters}] \leavevmode
\sphinxstyleliteralstrong{calcs} (\sphinxstyleliteralemphasis{dict}) -- a dictionary of dicionaries representing the set of calculations

\item[{Returns}] \leavevmode
None

\end{description}\end{quote}

\end{fulllineitems}

\index{modifyNamelistPW() (in module prep)}

\begin{fulllineitems}
\phantomsection\label{\detokenize{prep:prep.modifyNamelistPW}}\pysiglinewithargsret{\sphinxcode{prep.}\sphinxbfcode{modifyNamelistPW}}{\emph{calcs}, \emph{namelist}, \emph{parameter}, \emph{value}, \emph{runlocal=False}}{}
A Wrapper function that is used to write the function AFLOWpi.prep.\_modifyNameListPW
to the \_ID.py. If the value is intended to be a string in the QE input file, it must
be dually quoted i.e. value=```scf''' will become `scf' in the input file.
\begin{quote}\begin{description}
\item[{Parameters}] \leavevmode\begin{itemize}
\item {} 
\sphinxstyleliteralstrong{calcs} (\sphinxstyleliteralemphasis{dict}) -- a dictionary of dicionaries representing the set of calculations

\item {} 
\sphinxstyleliteralstrong{namelist} (\sphinxstyleliteralemphasis{str}) -- a string of the fortran namelist that the parameter is in

\item {} 
\sphinxstyleliteralstrong{parameter} (\sphinxstyleliteralemphasis{str}) -- a string of the parameter name

\item {} 
\sphinxstyleliteralstrong{value} -- the value of that parameter

\end{itemize}

\item[{Keyword Arguments}] \leavevmode
\sphinxstyleliteralstrong{runlocal} (\sphinxstyleliteralemphasis{bool}) -- a flag to choose whether or not to run the wrapped function now
or write it to the \_ID.py to run during the workflow.

\item[{Returns}] \leavevmode
Either the identical set of calculations if runlocal == False or the set of
calculations with the parameter's value changed in their oneCalc{[}'\_AFLOWPI\_INPUT\_'{]}
if runlocal==True

\end{description}\end{quote}

\end{fulllineitems}

\index{newstepWrapper() (in module prep)}

\begin{fulllineitems}
\phantomsection\label{\detokenize{prep:prep.newstepWrapper}}\pysiglinewithargsret{\sphinxcode{prep.}\sphinxbfcode{newstepWrapper}}{\emph{pre}}{}
A function that wraps another function that is to be run before the
a certain function runs. Its use must be in the form:

@newstepwrapper(func)
def being\_wrapped(oneCalc,ID,*args,**kwargs)

where func is the function that is to be run before and being\_wrapped
is the function being wrapped. oneCalc and ID must be the first two
arguments in the function being wrapped. additional arguments and
keyword arguments can follow.
\begin{quote}\begin{description}
\item[{Parameters}] \leavevmode
\sphinxstyleliteralstrong{pre} (\sphinxstyleliteralemphasis{func}) -- function object that is to be wrapped before another function runs

\item[{Returns}] \leavevmode
the returned values of function being wrapped or the execution of the function
is skipped entirely.

\end{description}\end{quote}

\end{fulllineitems}

\index{plotter (class in prep)}

\begin{fulllineitems}
\phantomsection\label{\detokenize{prep:prep.plotter}}\pysiglinewithargsret{\sphinxstrong{class }\sphinxcode{prep.}\sphinxbfcode{plotter}}{\emph{calcs}}{}
Class for adding common plotting functions from AFLOWpi.plot module to the high level user
interface.
\index{bands() (prep.plotter method)}

\begin{fulllineitems}
\phantomsection\label{\detokenize{prep:prep.plotter.bands}}\pysiglinewithargsret{\sphinxbfcode{bands}}{\emph{yLim={[}-10, 10{]}, DOSPlot='`, runlocal=False, postfix='`}}{}
Wrapper method to call AFLOWpi.plot.bands in the high level user interface.
\begin{quote}\begin{description}
\item[{Parameters}] \leavevmode
\sphinxstyleliteralstrong{self} -- the plotter object

\item[{Keyword Arguments}] \leavevmode\begin{itemize}
\item {} 
\sphinxstyleliteralstrong{yLim} (\sphinxstyleliteralemphasis{list}) -- a tuple or list of the range of energy around the fermi/Highest
occupied level energy that is to be included in the plot.

\item {} 
\sphinxstyleliteralstrong{DOSPlot} (\sphinxstyleliteralemphasis{str}) -- 
a string that flags for the option to have either a DOS plot
share the Y-axis of the band structure plot.

Options include:
``''      \textbar{} A blank string (default) will cause No Density of
\begin{quote}

\begin{DUlineblock}{0em}
\item[] States plotted alongside the Band Structure
\end{DUlineblock}
\end{quote}

``APDOS'' \textbar{} Atom Projected Density of States
``DOS''   \textbar{} Normal Density of States


\item {} 
\sphinxstyleliteralstrong{LSDA} (\sphinxstyleliteralemphasis{bool}) -- Plot the up and down of a spin polarized orbital projected DOS
calculation.

\item {} 
\sphinxstyleliteralstrong{runlocal} (\sphinxstyleliteralemphasis{bool}) -- a flag to choose whether or not to run the wrapped function now
or write it to the \_ID.py to run during the workflow

\item {} 
\sphinxstyleliteralstrong{postfix} (\sphinxstyleliteralemphasis{str}) -- a string of an optional postfix to the plot filename for every
calculation.

\end{itemize}

\item[{Returns}] \leavevmode
None

\end{description}\end{quote}

\end{fulllineitems}

\index{dos() (prep.plotter method)}

\begin{fulllineitems}
\phantomsection\label{\detokenize{prep:prep.plotter.dos}}\pysiglinewithargsret{\sphinxbfcode{dos}}{\emph{yLim={[}-10, 10{]}, runlocal=False, postfix='`}}{}
Wrapper method to call AFLOWpi.plot.dos in the high level user interface.
\begin{quote}\begin{description}
\item[{Parameters}] \leavevmode
\sphinxstyleliteralstrong{self} -- the plotter object

\item[{Keyword Arguments}] \leavevmode\begin{itemize}
\item {} 
\sphinxstyleliteralstrong{yLim} (\sphinxstyleliteralemphasis{list}) -- a tuple or list of the range of energy around the fermi/Highest
occupied level energy that is to be included in the plot.

\item {} 
\sphinxstyleliteralstrong{LSDA} (\sphinxstyleliteralemphasis{bool}) -- Plot the up and down of a spin polarized DOS
calculation.

\item {} 
\sphinxstyleliteralstrong{runlocal} (\sphinxstyleliteralemphasis{bool}) -- a flag to choose whether or not to run the wrapped function now
or write it to the \_ID.py to run during the workflow

\item {} 
\sphinxstyleliteralstrong{postfix} (\sphinxstyleliteralemphasis{str}) -- a string of an optional postfix to the plot filename for every
calculation.

\end{itemize}

\item[{Returns}] \leavevmode
None

\end{description}\end{quote}

\end{fulllineitems}

\index{opdos() (prep.plotter method)}

\begin{fulllineitems}
\phantomsection\label{\detokenize{prep:prep.plotter.opdos}}\pysiglinewithargsret{\sphinxbfcode{opdos}}{\emph{yLim={[}-10, 10{]}, runlocal=False, postfix='`}}{}
Wrapper method to call AFLOWpi.plot.opdos in the high level user interface.
\begin{quote}\begin{description}
\item[{Parameters}] \leavevmode
\sphinxstyleliteralstrong{self} -- the plotter object

\item[{Keyword Arguments}] \leavevmode\begin{itemize}
\item {} 
\sphinxstyleliteralstrong{yLim} (\sphinxstyleliteralemphasis{list}) -- a tuple or list of the range of energy around the fermi/Highest
occupied level energy that is to be included in the plot.

\item {} 
\sphinxstyleliteralstrong{LSDA} (\sphinxstyleliteralemphasis{bool}) -- Plot the up and down of a spin polarized orbital projected DOS
calculation.

\item {} 
\sphinxstyleliteralstrong{runlocal} (\sphinxstyleliteralemphasis{bool}) -- a flag to choose whether or not to run the wrapped function now
or write it to the \_ID.py to run during the workflow

\item {} 
\sphinxstyleliteralstrong{postfix} (\sphinxstyleliteralemphasis{string}) -- a string of an optional postfix to the plot filename for every
calculation.

\end{itemize}

\item[{Returns}] \leavevmode
None

\end{description}\end{quote}

\end{fulllineitems}

\index{phonon() (prep.plotter method)}

\begin{fulllineitems}
\phantomsection\label{\detokenize{prep:prep.plotter.phonon}}\pysiglinewithargsret{\sphinxbfcode{phonon}}{\emph{runlocal=False}, \emph{postfix='`}, \emph{THz=True}}{}
\end{fulllineitems}


\end{fulllineitems}

\index{prep\_split\_step() (in module prep)}

\begin{fulllineitems}
\phantomsection\label{\detokenize{prep:prep.prep_split_step}}\pysiglinewithargsret{\sphinxcode{prep.}\sphinxbfcode{prep\_split\_step}}{\emph{calcs}, \emph{subset\_creator}, \emph{subset\_tasks={[}{]}}, \emph{mult\_jobs=False}, \emph{substep\_name='SUBSET'}, \emph{keep\_file\_names=False}, \emph{clean\_input=True}, \emph{check\_function=None}, \emph{fault\_tolerant=False}}{}
\end{fulllineitems}

\index{remove\_blank\_lines() (in module prep)}

\begin{fulllineitems}
\phantomsection\label{\detokenize{prep:prep.remove_blank_lines}}\pysiglinewithargsret{\sphinxcode{prep.}\sphinxbfcode{remove\_blank\_lines}}{\emph{inp\_str}}{}
Removes whitespace lines of text
\begin{quote}\begin{description}
\item[{Parameters}] \leavevmode
\sphinxstyleliteralstrong{inp\_str} (\sphinxstyleliteralemphasis{str}) -- input string of text

\item[{Returns}] \leavevmode
the same string with blank lines removed

\end{description}\end{quote}

\end{fulllineitems}

\index{runAfterAllDone() (in module prep)}

\begin{fulllineitems}
\phantomsection\label{\detokenize{prep:prep.runAfterAllDone}}\pysiglinewithargsret{\sphinxcode{prep.}\sphinxbfcode{runAfterAllDone}}{\emph{calcs}, \emph{command}, \emph{faultTolerant=True}}{}
Adds a command to the BATCH command block at the end of each calculation's \_ID.py
for all calculations in the set. Used to execute a command over all calculations
in particular step have completed.
\begin{quote}\begin{description}
\item[{Parameters}] \leavevmode
\sphinxstyleliteralstrong{calcs} (\sphinxstyleliteralemphasis{dict}) -- a dictionary of dicionaries representing the set of calculations

\item[{Keyword Arguments}] \leavevmode\begin{itemize}
\item {} 
\sphinxstyleliteralstrong{command} (\sphinxstyleliteralemphasis{str}) -- the text to be added to the BATCH block

\item {} 
\sphinxstyleliteralstrong{faultTolerant} (\sphinxstyleliteralemphasis{bool}) -- a flag to choose if we return True if some of the
calculations ran but did not complete successful

\end{itemize}

\item[{Returns}] \leavevmode
None

\end{description}\end{quote}

\end{fulllineitems}

\index{scfs() (in module prep)}

\begin{fulllineitems}
\phantomsection\label{\detokenize{prep:prep.scfs}}\pysiglinewithargsret{\sphinxcode{prep.}\sphinxbfcode{scfs}}{\emph{aflowkeys}, \emph{allAFLOWpiVars}, \emph{refFile}, \emph{pseudodir=None}, \emph{build\_type='product'}, \emph{convert=False}}{}
Read a reference input file, and construct a set of calculations from the allAFLOWpiVars
dictionary defining values for the keywords in the reference input file. This will
also create directory within the set directory for every calculation in the set.
\begin{quote}\begin{description}
\item[{Parameters}] \leavevmode\begin{itemize}
\item {} 
\sphinxstyleliteralstrong{allAFLOWpiVars} (\sphinxstyleliteralemphasis{dict}) -- a dictionary whose keys correspond to the keywords in the
reference input file and whose values will be used to
construct the set of calculations

\item {} 
\sphinxstyleliteralstrong{refFile} (\sphinxstyleliteralemphasis{str}) -- a filename as a string, a file object, or a string of the file
that contains keywords to construct the inputs to the different
calculations in the set

\end{itemize}

\item[{Keyword Arguments}] \leavevmode\begin{itemize}
\item {} 
\sphinxstyleliteralstrong{pseudodir} (\sphinxstyleliteralemphasis{str}) -- path of the directory that contains your Pseudopotential files
The value in the AFLOWpi config file used will override this.

\item {} 
\sphinxstyleliteralstrong{build\_type} (\sphinxstyleliteralemphasis{str}) -- 
how to construct the calculation set from allAFLOWpiVars dictionary:
\begin{description}
\item[{zip \textbar{} The first calculation takes the first entry from the list of}] \leavevmode
\begin{DUlineblock}{0em}
\item[] each of the keywords. The second calculation takes the second
\item[] and so on. The keywords for all lists in allAFLOWpiVars must be
\item[] the same length for this method.
\end{DUlineblock}

\item[{product \textbar{} Calculation set is formed via a ``cartesian product'' with}] \leavevmode
\begin{DUlineblock}{0em}
\item[] the values the list of each keyword combined. (i.e if
\item[] allAFLOWpiVars has one keyword with a list of 5 entires and
\item[] another with 4 and a third with 10, there would be 2000
\item[] calculations in the set formed from them via product mode.
\end{DUlineblock}

\end{description}


\end{itemize}

\item[{Returns}] \leavevmode
A dictionary of dictionaries containing the set of calculations.

\end{description}\end{quote}

\end{fulllineitems}

\index{tb\_plotter (class in prep)}

\begin{fulllineitems}
\phantomsection\label{\detokenize{prep:prep.tb_plotter}}\pysiglinewithargsret{\sphinxstrong{class }\sphinxcode{prep.}\sphinxbfcode{tb\_plotter}}{\emph{calcs}}{}
Class for adding common plotting functions from AFLOWpi.plot module to the high level user
interface.
\index{bands() (prep.tb\_plotter method)}

\begin{fulllineitems}
\phantomsection\label{\detokenize{prep:prep.tb_plotter.bands}}\pysiglinewithargsret{\sphinxbfcode{bands}}{\emph{yLim={[}-5, 5{]}, DOSPlot='`, runlocal=False, postfix='`}}{}
\end{fulllineitems}

\index{dos() (prep.tb\_plotter method)}

\begin{fulllineitems}
\phantomsection\label{\detokenize{prep:prep.tb_plotter.dos}}\pysiglinewithargsret{\sphinxbfcode{dos}}{\emph{yLim={[}-5, 5{]}, runlocal=False, postfix='`}}{}
\end{fulllineitems}

\index{opdos() (prep.tb\_plotter method)}

\begin{fulllineitems}
\phantomsection\label{\detokenize{prep:prep.tb_plotter.opdos}}\pysiglinewithargsret{\sphinxbfcode{opdos}}{\emph{yLim={[}-5, 5{]}, runlocal=False, postfix='`}}{}
\end{fulllineitems}

\index{optical() (prep.tb\_plotter method)}

\begin{fulllineitems}
\phantomsection\label{\detokenize{prep:prep.tb_plotter.optical}}\pysiglinewithargsret{\sphinxbfcode{optical}}{\emph{runlocal=False}, \emph{postfix='`}, \emph{x\_range=None}}{}
Wrapper method to call AFLOWpi.plot.epsilon in the high level user interface.
\begin{quote}\begin{description}
\item[{Parameters}] \leavevmode
\sphinxstyleliteralstrong{self} -- the plotter object

\item[{Keyword Arguments}] \leavevmode\begin{itemize}
\item {} 
\sphinxstyleliteralstrong{nm} (\sphinxstyleliteralemphasis{bool}) -- whether to plot in nanometers for spectrum or eV for energy

\item {} 
\sphinxstyleliteralstrong{runlocal} (\sphinxstyleliteralemphasis{bool}) -- a flag to choose whether or not to run the wrapped function now
or write it to the \_ID.py to run during the workflow

\end{itemize}

\item[{Returns}] \leavevmode
None

\end{description}\end{quote}

\end{fulllineitems}

\index{transport() (prep.tb\_plotter method)}

\begin{fulllineitems}
\phantomsection\label{\detokenize{prep:prep.tb_plotter.transport}}\pysiglinewithargsret{\sphinxbfcode{transport}}{\emph{runlocal=False}, \emph{postfix='`}, \emph{x\_range=None}}{}
Wrapper method to call AFLOWpi.plot.epsilon in the high level user interface.
\begin{quote}\begin{description}
\item[{Parameters}] \leavevmode
\sphinxstyleliteralstrong{self} -- the plotter object

\item[{Keyword Arguments}] \leavevmode\begin{itemize}
\item {} 
\sphinxstyleliteralstrong{nm} (\sphinxstyleliteralemphasis{bool}) -- whether to plot in nanometers for spectrum or eV for energy

\item {} 
\sphinxstyleliteralstrong{runlocal} (\sphinxstyleliteralemphasis{bool}) -- a flag to choose whether or not to run the wrapped function now
or write it to the \_ID.py to run during the workflow

\end{itemize}

\item[{Returns}] \leavevmode
None

\end{description}\end{quote}

\end{fulllineitems}


\end{fulllineitems}

\index{tight\_binding (class in prep)}

\begin{fulllineitems}
\phantomsection\label{\detokenize{prep:prep.tight_binding}}\pysiglinewithargsret{\sphinxstrong{class }\sphinxcode{prep.}\sphinxbfcode{tight\_binding}}{\emph{calcs}, \emph{cond\_bands=True}, \emph{proj\_thr=0.95}, \emph{kp\_factor=2.0}, \emph{proj\_sh=5.5}, \emph{cond\_bands\_proj=True}}{}~\index{bands() (prep.tight\_binding method)}

\begin{fulllineitems}
\phantomsection\label{\detokenize{prep:prep.tight_binding.bands}}\pysiglinewithargsret{\sphinxbfcode{bands}}{\emph{nk=1000}, \emph{nbnd=None}, \emph{eShift=15.0}, \emph{cond\_bands=True}}{}
\end{fulllineitems}

\index{dos() (prep.tight\_binding method)}

\begin{fulllineitems}
\phantomsection\label{\detokenize{prep:prep.tight_binding.dos}}\pysiglinewithargsret{\sphinxbfcode{dos}}{\emph{dos\_range={[}-5.5, 5.5{]}, k\_grid=None, projected=True, de=0.05, cond\_bands=True, fermi\_surface=False}}{}
\end{fulllineitems}

\index{effmass() (prep.tight\_binding method)}

\begin{fulllineitems}
\phantomsection\label{\detokenize{prep:prep.tight_binding.effmass}}\pysiglinewithargsret{\sphinxbfcode{effmass}}{\emph{temperature={[}300.0, 300.0{]}, step=1}}{}
\end{fulllineitems}

\index{optical() (prep.tight\_binding method)}

\begin{fulllineitems}
\phantomsection\label{\detokenize{prep:prep.tight_binding.optical}}\pysiglinewithargsret{\sphinxbfcode{optical}}{\emph{en\_range={[}0.05, 5.05{]}, de=0.05}}{}
\end{fulllineitems}

\index{transport() (prep.tight\_binding method)}

\begin{fulllineitems}
\phantomsection\label{\detokenize{prep:prep.tight_binding.transport}}\pysiglinewithargsret{\sphinxbfcode{transport}}{\emph{temperature={[}300{]}, en\_range={[}-5.05, 5.05{]}, de=0.05}}{}
Wrapper method to call AFLOWpi.scfuj.prep\_transport and AFLOWpi.scfuj.run\_transport
in the high level user interface. Adds a new step to the workflow.
\begin{quote}\begin{description}
\item[{Parameters}] \leavevmode
\sphinxstyleliteralstrong{self} -- the \_calcs\_container object

\item[{Keyword Arguments}] \leavevmode\begin{itemize}
\item {} 
\sphinxstyleliteralstrong{epsilon} (\sphinxstyleliteralemphasis{bool}) -- if True episilon tensor will be computed

\item {} 
\sphinxstyleliteralstrong{temperature} (\sphinxstyleliteralemphasis{list}) -- list of temperature(s) at which to calculate transport properties

\end{itemize}

\item[{Returns}] \leavevmode
None

\end{description}\end{quote}

\end{fulllineitems}


\end{fulllineitems}

\index{totree() (in module prep)}

\begin{fulllineitems}
\phantomsection\label{\detokenize{prep:prep.totree}}\pysiglinewithargsret{\sphinxcode{prep.}\sphinxbfcode{totree}}{\emph{tobecopied}, \emph{calcs}, \emph{rename=None}, \emph{symlink=False}}{}
Populate all the subdirectories for the calculation with the file in input
\begin{quote}\begin{description}
\item[{Parameters}] \leavevmode\begin{itemize}
\item {} 
\sphinxstyleliteralstrong{tobecopied} (\sphinxstyleliteralemphasis{str}) -- filepath to be copied to the AFLOWpi directory tree

\item {} 
\sphinxstyleliteralstrong{calcs} (\sphinxstyleliteralemphasis{dict}) -- Dictionary of dictionaries of calculations

\end{itemize}

\item[{Keyword Arguments}] \leavevmode\begin{itemize}
\item {} 
\sphinxstyleliteralstrong{rename} (\sphinxstyleliteralemphasis{bool}) -- option to rename the file/directory being moves into the AFLOWpi
directory tree

\item {} 
\sphinxstyleliteralstrong{symlink} (\sphinxstyleliteralemphasis{bool}) -- whether to copy the data to the AFLOWpi directory tree or
to use symbolic links

\end{itemize}

\item[{Returns}] \leavevmode
None

\end{description}\end{quote}

\end{fulllineitems}

\index{unlockAtomMovement() (in module prep)}

\begin{fulllineitems}
\phantomsection\label{\detokenize{prep:prep.unlockAtomMovement}}\pysiglinewithargsret{\sphinxcode{prep.}\sphinxbfcode{unlockAtomMovement}}{\emph{calcs}}{}
A Wrapper function that writes the function AFLOWpi.prep.\_unfreezeAtoms to the \_ID.py
\begin{quote}\begin{description}
\item[{Parameters}] \leavevmode
\sphinxstyleliteralstrong{calcs} (\sphinxstyleliteralemphasis{dict}) -- a dictionary of dicionaries representing the set of calculations

\item[{Returns}] \leavevmode
None

\end{description}\end{quote}

\end{fulllineitems}

\index{updateStructs() (in module prep)}

\begin{fulllineitems}
\phantomsection\label{\detokenize{prep:prep.updateStructs}}\pysiglinewithargsret{\sphinxcode{prep.}\sphinxbfcode{updateStructs}}{\emph{calcs}, \emph{update\_structure=True}, \emph{update\_positions=True}}{}
A Wrapper function that writes the function AFLOWpi.prep.\_oneUpdateStructs to the \_ID.py
\begin{quote}\begin{description}
\item[{Parameters}] \leavevmode
\sphinxstyleliteralstrong{calcs} (\sphinxstyleliteralemphasis{dict}) -- a dictionary of dicionaries representing the set of calculations

\item[{Keyword Arguments}] \leavevmode\begin{itemize}
\item {} 
\sphinxstyleliteralstrong{update\_structure} (\sphinxstyleliteralemphasis{bool}) -- if True update the cell parameter if possible from the
output of previous calculations in the workflow.

\item {} 
\sphinxstyleliteralstrong{update\_positions} (\sphinxstyleliteralemphasis{bool}) -- if True update the atomic positions if possible from the
output of previous calculations in the workflow.

\end{itemize}

\item[{Returns}] \leavevmode
The identical set of calculations as the input to this function

\end{description}\end{quote}

\end{fulllineitems}

\index{updatelogs() (in module prep)}

\begin{fulllineitems}
\phantomsection\label{\detokenize{prep:prep.updatelogs}}\pysiglinewithargsret{\sphinxcode{prep.}\sphinxbfcode{updatelogs}}{\emph{calcs}, \emph{logname}, \emph{runlocal=False}}{}
\end{fulllineitems}

\index{varyCellParams() (in module prep)}

\begin{fulllineitems}
\phantomsection\label{\detokenize{prep:prep.varyCellParams}}\pysiglinewithargsret{\sphinxcode{prep.}\sphinxbfcode{varyCellParams}}{\emph{oneCalc}, \emph{ID}, \emph{param=()}, \emph{amount=0.15}, \emph{steps=8}, \emph{constraint=None}}{}
Forms and returns a set of calcs with varied cell params must be in A,B,C,
and, in degrees,alpha,beta,gamma and then returns it.
\begin{quote}\begin{description}
\item[{Parameters}] \leavevmode\begin{itemize}
\item {} 
\sphinxstyleliteralstrong{oneCalc} (\sphinxstyleliteralemphasis{dict}) -- a dictionary containing properties about the AFLOWpi calculation

\item {} 
\sphinxstyleliteralstrong{ID} (\sphinxstyleliteralemphasis{str}) -- ID string for the particular calculation and step

\end{itemize}

\item[{Keyword Arguments}] \leavevmode\begin{itemize}
\item {} 
\sphinxstyleliteralstrong{param} (\sphinxstyleliteralemphasis{tuple}) -- the params assoc. with the amount and step. i.e. (`celldm(1)','celldm(3))

\item {} 
\sphinxstyleliteralstrong{amount} (\sphinxstyleliteralemphasis{float}) -- percentage amount to be varied up and down. i.e (0.04,0.02,0.01)

\item {} 
\sphinxstyleliteralstrong{steps} (\sphinxstyleliteralemphasis{int}) -- how many steps to within each range. i.e (4,5,7)

\item {} 
\sphinxstyleliteralstrong{constraint} (\sphinxstyleliteralemphasis{list}) -- a list or tuple containing two entry long list or tuples with
the first being the constraint type and the second the free
parameter in params that its constraining for example in a
orthorhombic cell: constraint=({[}''volume'','c'{]},) allows for A and B
to move freely but C is such that it keeps the cell volume the same
in all calculations generated by the input oneCalc calculation.

\end{itemize}

\item[{Returns}] \leavevmode
A dictionary of dictionaries representing a new calculation set

\end{description}\end{quote}

\end{fulllineitems}

\index{writeToScript() (in module prep)}

\begin{fulllineitems}
\phantomsection\label{\detokenize{prep:prep.writeToScript}}\pysiglinewithargsret{\sphinxcode{prep.}\sphinxbfcode{writeToScript}}{\emph{executable}, \emph{calcs}, \emph{from\_step=0}}{}
Generates calls on several functions to set up everything that is needed for a new step in the workflow.
The mechanics of the \_ID.py are written to it here.
\begin{quote}\begin{description}
\item[{Parameters}] \leavevmode\begin{itemize}
\item {} 
\sphinxstyleliteralstrong{calcs} (\sphinxstyleliteralemphasis{dict}) -- dictionary of dictionaries of calculations

\item {} 
\sphinxstyleliteralstrong{executable} (\sphinxstyleliteralemphasis{str}) -- \textless{}DEFUNCT OPTION: HERE FOR LEGACY SUPPORT\textgreater{}

\item {} 
\sphinxstyleliteralstrong{*args} -- \textless{}DEFUNCT OPTION: HERE FOR LEGACY SUPPORT\textgreater{}

\end{itemize}

\item[{Keyword Arguments}] \leavevmode
\sphinxstyleliteralstrong{**kwargs} -- \textless{}DEFUNCT OPTION: HERE FOR LEGACY SUPPORT\textgreater{}

\item[{Returns}] \leavevmode
A set of calculations for a new step in the workflow

\end{description}\end{quote}

\end{fulllineitems}



\chapter{pseudo module}
\label{\detokenize{pseudo::doc}}\label{\detokenize{pseudo:pseudo-module}}\label{\detokenize{pseudo:module-pseudo}}\index{pseudo (module)}\index{brute\_test() (in module pseudo)}

\begin{fulllineitems}
\phantomsection\label{\detokenize{pseudo:pseudo.brute_test}}\pysiglinewithargsret{\sphinxcode{pseudo.}\sphinxbfcode{brute\_test}}{\emph{calcs}, \emph{ecutwfc}, \emph{dual=None}, \emph{sampling=None}, \emph{constraint=None}, \emph{thresh=0.001}, \emph{initial\_variance=0.05}, \emph{grid\_density=10}, \emph{mult\_jobs=False}, \emph{conv\_thresh=0.01}, \emph{calc\_type='relax'}}{}
\end{fulllineitems}

\index{crawlingMinimization() (in module pseudo)}

\begin{fulllineitems}
\phantomsection\label{\detokenize{pseudo:pseudo.crawlingMinimization}}\pysiglinewithargsret{\sphinxcode{pseudo.}\sphinxbfcode{crawlingMinimization}}{\emph{calcs}, \emph{options=None}, \emph{faultTolerant=True}, \emph{constraint=None}, \emph{thresh=0.001}, \emph{initial\_variance=0.05}, \emph{grid\_density=10}, \emph{mult\_jobs=False}, \emph{final\_minimization='relax'}, \emph{calc\_type='relax'}}{}
\end{fulllineitems}

\index{getMinimization() (in module pseudo)}

\begin{fulllineitems}
\phantomsection\label{\detokenize{pseudo:pseudo.getMinimization}}\pysiglinewithargsret{\sphinxcode{pseudo.}\sphinxbfcode{getMinimization}}{\emph{origCalcs}, \emph{fitVars=None}, \emph{options=None}, \emph{runlocal=False}, \emph{faultTolerant=True}, \emph{minimize\_var='Energy'}}{}
\end{fulllineitems}

\index{plot() (in module pseudo)}

\begin{fulllineitems}
\phantomsection\label{\detokenize{pseudo:pseudo.plot}}\pysiglinewithargsret{\sphinxcode{pseudo.}\sphinxbfcode{plot}}{\emph{resultList}, \emph{xaxis='`}, \emph{xtitle=None}, \emph{ytitle=None}, \emph{title=None}, \emph{rename=None}, \emph{file\_name='`}}{}
takes in a list of dictionaries of dictionaries of calculations and generates
a plot with the x axis being some value in the list `key' and splits the calculations
and plots them with each plot being a unique combination of the items in key that are not
`xaxis'
\begin{quote}\begin{description}
\item[{Parameters}] \leavevmode\begin{itemize}
\item {} 
\sphinxstyleliteralstrong{resultList} (\sphinxstyleliteralemphasis{list}) -- list of dictionaries of dictionaries of calculations

\item {} 
\sphinxstyleliteralstrong{xaxis} (\sphinxstyleliteralemphasis{str}) -- the keyword in oneCalc that you choose to be the x axis in your plots

\end{itemize}

\item[{Keyword Arguments}] \leavevmode\begin{itemize}
\item {} 
\sphinxstyleliteralstrong{xtitle} (\sphinxstyleliteralemphasis{str}) -- title of the x axis of the plot (default: None)

\item {} 
\sphinxstyleliteralstrong{ytitle} (\sphinxstyleliteralemphasis{str}) -- title of the y axis of the plot (default: None)

\item {} 
\sphinxstyleliteralstrong{plotTitle} (\sphinxstyleliteralemphasis{str}) -- title of the plots (default: None)

\item {} 
\sphinxstyleliteralstrong{rename} (\sphinxstyleliteralemphasis{dict}) -- a mapping of the names of the keywords of whose
values used to generate the plot to some other name.
ex. \{` \_AFLOWPI\_ECUTW\_':'wavefunction cutoff'\}

\item {} 
\sphinxstyleliteralstrong{file\_name} (\sphinxstyleliteralemphasis{str}) -- use this instead of ``PT\_RESULTS.pdf'' as filename of plot

\end{itemize}

\item[{Returns}] \leavevmode
None

\end{description}\end{quote}

\end{fulllineitems}



\chapter{retr module}
\label{\detokenize{retr::doc}}\label{\detokenize{retr:retr-module}}\phantomsection\label{\detokenize{retr:module-retr}}\index{retr (module)}\index{aflow\_prim2conv() (in module retr)}

\begin{fulllineitems}
\phantomsection\label{\detokenize{retr:retr.aflow_prim2conv}}\pysiglinewithargsret{\sphinxcode{retr.}\sphinxbfcode{aflow\_prim2conv}}{\emph{*args}, \emph{**kwargs}}{}
Converts primitive cell vectors to conventional cell vectors using AFLOW
convention.
\begin{quote}\begin{description}
\item[{Parameters}] \leavevmode\begin{itemize}
\item {} 
\sphinxstyleliteralstrong{cell} (\sphinxstyleliteralemphasis{numpy.matrix}) -- 3x3 AFLOW primitive cell vectors

\item {} 
\sphinxstyleliteralstrong{ibrav} (\sphinxstyleliteralemphasis{int}) -- Quantum Espresso Bravais lattice index of the crystal
structure

\end{itemize}

\item[{Returns}] \leavevmode
3x3 \sphinxtitleref{numpy.matrix} of the cell in AFLOW conventional lattice vectors.

\end{description}\end{quote}

\end{fulllineitems}

\index{aflow\_conv2prim() (in module retr)}

\begin{fulllineitems}
\phantomsection\label{\detokenize{retr:retr.aflow_conv2prim}}\pysiglinewithargsret{\sphinxcode{retr.}\sphinxbfcode{aflow\_conv2prim}}{\emph{*args}, \emph{**kwargs}}{}
Converts conventional cell vectors to primitive cell vectors using AFLOW
convention.
\begin{quote}\begin{description}
\item[{Parameters}] \leavevmode\begin{itemize}
\item {} 
\sphinxstyleliteralstrong{cell} (\sphinxstyleliteralemphasis{numpy.matrix}) -- 3x3 AFLOW conventional cell vectors

\item {} 
\sphinxstyleliteralstrong{ibrav} (\sphinxstyleliteralemphasis{int}) -- Quantum Espresso Bravais lattice index of the crystal
structure

\end{itemize}

\item[{Returns}] \leavevmode
3x3 \sphinxtitleref{numpy.matrix} of the cell in AFLOW primitive lattice vectors.

\end{description}\end{quote}

\end{fulllineitems}

\index{qe\_prim2conv() (in module retr)}

\begin{fulllineitems}
\phantomsection\label{\detokenize{retr:retr.qe_prim2conv}}\pysiglinewithargsret{\sphinxcode{retr.}\sphinxbfcode{qe\_prim2conv}}{\emph{*args}, \emph{**kwargs}}{}
Converts primitive cell vectors to conventional cell vectors using Quantum
Espresso convention.
\begin{quote}\begin{description}
\item[{Parameters}] \leavevmode\begin{itemize}
\item {} 
\sphinxstyleliteralstrong{cell} (\sphinxstyleliteralemphasis{numpy.matrix}) -- 3x3 Quantum Espresso conventional cell vectors

\item {} 
\sphinxstyleliteralstrong{ibrav} (\sphinxstyleliteralemphasis{int}) -- Quantum Espresso Bravais lattice index of the crystal
structure

\end{itemize}

\item[{Returns}] \leavevmode
3x3 \sphinxtitleref{numpy.matrix} of the cell in Quantum Espresso primitive lattice
vectors.

\end{description}\end{quote}

\end{fulllineitems}

\index{qe\_conv2prim() (in module retr)}

\begin{fulllineitems}
\phantomsection\label{\detokenize{retr:retr.qe_conv2prim}}\pysiglinewithargsret{\sphinxcode{retr.}\sphinxbfcode{qe\_conv2prim}}{\emph{*args}, \emph{**kwargs}}{}
Converts conventional cell vectors to primitive cell vectors using Quantum
Espresso convention.
\begin{quote}\begin{description}
\item[{Parameters}] \leavevmode\begin{itemize}
\item {} 
\sphinxstyleliteralstrong{cell} (\sphinxstyleliteralemphasis{numpy.matrix}) -- 3x3 Quantum Espresso conventional cell vectors

\item {} 
\sphinxstyleliteralstrong{ibrav} (\sphinxstyleliteralemphasis{int}) -- Quantum Espresso Bravais lattice index of the crystal
structure

\end{itemize}

\item[{Returns}] \leavevmode
3x3 \sphinxtitleref{numpy.matrix} of the cell in Quantum Espresso primitive lattice
vectors.

\end{description}\end{quote}

\end{fulllineitems}

\index{conv\_aflow2qe() (in module retr)}

\begin{fulllineitems}
\phantomsection\label{\detokenize{retr:retr.conv_aflow2qe}}\pysiglinewithargsret{\sphinxcode{retr.}\sphinxbfcode{conv\_aflow2qe}}{\emph{*args}, \emph{**kwargs}}{}
Converts conventional cell vectors using AFLOW convention to conventional
cell vectors using Quantum Espresso convention.
\begin{quote}\begin{description}
\item[{Parameters}] \leavevmode\begin{itemize}
\item {} 
\sphinxstyleliteralstrong{cell} (\sphinxstyleliteralemphasis{numpy.matrix}) -- 3x3 AFLOW conventional cell vectors

\item {} 
\sphinxstyleliteralstrong{ibrav} (\sphinxstyleliteralemphasis{int}) -- Quantum Espresso Bravais lattice index of the crystal
structure

\item {} 
\sphinxstyleliteralstrong{angle} (\sphinxstyleliteralemphasis{float}) -- Characteristic angle for the lattice structure, only
necessary for rhombohedral (RHL) lattices

\end{itemize}

\item[{Returns}] \leavevmode
3x3 \sphinxtitleref{numpy.matrix} of the cell in Quantum Espresso conventional lattice
vectors.

\end{description}\end{quote}

\end{fulllineitems}

\index{conv\_qe2aflow() (in module retr)}

\begin{fulllineitems}
\phantomsection\label{\detokenize{retr:retr.conv_qe2aflow}}\pysiglinewithargsret{\sphinxcode{retr.}\sphinxbfcode{conv\_qe2aflow}}{\emph{*args}, \emph{**kwargs}}{}
Converts conventional cell vectors using Quantum Espresso convention to
conventional cell vectors using AFLOW convention.
\begin{quote}\begin{description}
\item[{Parameters}] \leavevmode\begin{itemize}
\item {} 
\sphinxstyleliteralstrong{cell} (\sphinxstyleliteralemphasis{numpy.matrix}) -- 3x3 Quantum Espresso conventional cell vectors

\item {} 
\sphinxstyleliteralstrong{ibrav} (\sphinxstyleliteralemphasis{int}) -- Quantum Espresso Bravais lattice index of the crystal
structure

\item {} 
\sphinxstyleliteralstrong{angle} (\sphinxstyleliteralemphasis{float}) -- Characteristic angle for the lattice structure, only
necessary for rhombohedral (RHL) lattices

\end{itemize}

\item[{Returns}] \leavevmode
3x3 \sphinxtitleref{numpy.matrix} of the cell in AFLOW conventional lattice vectors.

\end{description}\end{quote}

\end{fulllineitems}

\index{prim\_aflow2qe() (in module retr)}

\begin{fulllineitems}
\phantomsection\label{\detokenize{retr:retr.prim_aflow2qe}}\pysiglinewithargsret{\sphinxcode{retr.}\sphinxbfcode{prim\_aflow2qe}}{\emph{*args}, \emph{**kwargs}}{}
Converts primitive cell vectors using AFLOW convention to primitive cell
vectors using Quantum Espresso convention.
\begin{quote}\begin{description}
\item[{Parameters}] \leavevmode\begin{itemize}
\item {} 
\sphinxstyleliteralstrong{cell} (\sphinxstyleliteralemphasis{numpy.matrix}) -- 3x3 AFLOW primitive cell vectors

\item {} 
\sphinxstyleliteralstrong{ibrav} (\sphinxstyleliteralemphasis{int}) -- Quantum Espresso Bravais lattice index of the crystal
structure

\end{itemize}

\item[{Returns}] \leavevmode
3x3 \sphinxtitleref{numpy.matrix} of the cell in Quantum Espresso primitive lattice
vectors.

\end{description}\end{quote}

\end{fulllineitems}

\index{prim\_qe2aflow() (in module retr)}

\begin{fulllineitems}
\phantomsection\label{\detokenize{retr:retr.prim_qe2aflow}}\pysiglinewithargsret{\sphinxcode{retr.}\sphinxbfcode{prim\_qe2aflow}}{\emph{*args}, \emph{**kwargs}}{}
Converts primitive cell vectors using Quantum Espresso convention to
primitive cell vectors using AFLOW convention.
\begin{quote}\begin{description}
\item[{Parameters}] \leavevmode\begin{itemize}
\item {} 
\sphinxstyleliteralstrong{cell} (\sphinxstyleliteralemphasis{numpy.matrix}) -- 3x3 AFLOW primitive cell vectors

\item {} 
\sphinxstyleliteralstrong{ibrav} (\sphinxstyleliteralemphasis{int}) -- Quantum Espresso Bravais lattice index of the crystal
structure

\end{itemize}

\item[{Returns}] \leavevmode
3x3 \sphinxtitleref{numpy.matrix} of the cell in Quantum Espresso primitive lattice
vectors.

\end{description}\end{quote}

\end{fulllineitems}



\chapter{run module}
\label{\detokenize{run:module-run}}\label{\detokenize{run::doc}}\label{\detokenize{run:run-module}}\index{run (module)}\index{addatexit\_\_() (in module run)}

\begin{fulllineitems}
\phantomsection\label{\detokenize{run:run.addatexit__}}\pysiglinewithargsret{\sphinxcode{run.}\sphinxbfcode{addatexit\_\_}}{\emph{command}, \emph{*args}, \emph{**kwargs}}{}
Wrapper to add function to be run at exit
\begin{quote}\begin{description}
\item[{Parameters}] \leavevmode\begin{itemize}
\item {} 
\sphinxstyleliteralstrong{command} (\sphinxstyleliteralemphasis{func}) -- function to be run

\item {} 
\sphinxstyleliteralstrong{*args} -- arguments for command

\end{itemize}

\item[{Keyword Arguments}] \leavevmode
\sphinxstyleliteralstrong{**kwargs} -- Keyword arguments for command

\item[{Returns}] \leavevmode
None

\end{description}\end{quote}

\end{fulllineitems}

\index{bands() (in module run)}

\begin{fulllineitems}
\phantomsection\label{\detokenize{run:run.bands}}\pysiglinewithargsret{\sphinxcode{run.}\sphinxbfcode{bands}}{\emph{calcs}, \emph{engine='`}, \emph{execPrefix=None}, \emph{execPostfix=' `}, \emph{holdFlag=True}, \emph{config=None}}{}
Wrapper to set up Electronic Band Structure calculation
\begin{quote}\begin{description}
\item[{Parameters}] \leavevmode
\sphinxstyleliteralstrong{calcs} (\sphinxstyleliteralemphasis{dict}) -- Dictionary of dictionaries of calculations

\item[{Keyword Arguments}] \leavevmode\begin{itemize}
\item {} 
\sphinxstyleliteralstrong{engine} (\sphinxstyleliteralemphasis{str}) -- executable that you are calling to run the calculations

\item {} 
\sphinxstyleliteralstrong{execPrefix} (\sphinxstyleliteralemphasis{str}) -- commands to go before the executable when run
(ex. mpiexec nice -n 19 \textless{}executable\textgreater{}) (default = None)

\item {} 
\sphinxstyleliteralstrong{execPostfix} (\sphinxstyleliteralemphasis{str}) -- commands to go after the executable when run
(ex. \textless{}execPrefix\textgreater{} \textless{}executable\textgreater{} -ndiag 12 -nimage 2) (default = None)

\item {} 
\sphinxstyleliteralstrong{holdFlag} (\sphinxstyleliteralemphasis{bool}) -- DEFUNCT. NEEDS REMOVAL

\item {} 
\sphinxstyleliteralstrong{config} (\sphinxstyleliteralemphasis{str}) -- DEFUNCT. NEEDS REMOVAL

\end{itemize}

\item[{Returns}] \leavevmode
None

\end{description}\end{quote}

\end{fulllineitems}

\index{clean\_cell\_params() (in module run)}

\begin{fulllineitems}
\phantomsection\label{\detokenize{run:run.clean_cell_params}}\pysiglinewithargsret{\sphinxcode{run.}\sphinxbfcode{clean\_cell\_params}}{\emph{output}}{}
Parses the atomic shifts in a supercell from fd.x
outputted pw.x input files to correct for formatting
issues when they are imported to AFLOWpi
\begin{quote}\begin{description}
\item[{Parameters}] \leavevmode
\sphinxstyleliteralstrong{output} (\sphinxstyleliteralemphasis{str}) -- pw.x input files generated by fd.x

\item[{Keyword Arguments}] \leavevmode
\sphinxstyleliteralstrong{None} -- 

\item[{Returns}] \leavevmode
pw.x input files generated by fd.x (cleaned by AFLOWpi)

\item[{Return type}] \leavevmode
output (str)

\end{description}\end{quote}

\end{fulllineitems}

\index{cleanup() (in module run)}

\begin{fulllineitems}
\phantomsection\label{\detokenize{run:run.cleanup}}\pysiglinewithargsret{\sphinxcode{run.}\sphinxbfcode{cleanup}}{\emph{calcs}}{}
Deletes all files a calculation set's  directory
tree that are prepended with `\_'
\begin{quote}\begin{description}
\item[{Parameters}] \leavevmode
\sphinxstyleliteralstrong{calcs} (\sphinxstyleliteralemphasis{dict}) -- Dictionary of dictionaries of calculations

\item[{Keyword Arguments}] \leavevmode
\sphinxstyleliteralstrong{None} -- 

\item[{Returns}] \leavevmode
None

\end{description}\end{quote}

\end{fulllineitems}

\index{dos() (in module run)}

\begin{fulllineitems}
\phantomsection\label{\detokenize{run:run.dos}}\pysiglinewithargsret{\sphinxcode{run.}\sphinxbfcode{dos}}{\emph{calcs}, \emph{engine='`}, \emph{execPrefix=None}, \emph{execPostfix=None}, \emph{holdFlag=True}, \emph{config=None}}{}
Wrapper to set up DOS nscf calculation
\begin{quote}\begin{description}
\item[{Parameters}] \leavevmode
\sphinxstyleliteralstrong{calcs} (\sphinxstyleliteralemphasis{dict}) -- Dictionary of dictionaries of calculations

\item[{Keyword Arguments}] \leavevmode\begin{itemize}
\item {} 
\sphinxstyleliteralstrong{engine} (\sphinxstyleliteralemphasis{str}) -- executable that you are calling to run the calculations

\item {} 
\sphinxstyleliteralstrong{execPrefix} (\sphinxstyleliteralemphasis{str}) -- commands to go before the executable when run
(ex. mpiexec nice -n 19 \textless{}executable\textgreater{}) (default = None)

\item {} 
\sphinxstyleliteralstrong{execPostfix} (\sphinxstyleliteralemphasis{str}) -- commands to go after the executable when run
(ex. \textless{}execPrefix\textgreater{} \textless{}executable\textgreater{} -ndiag 12 -nimage 2) (default = None)

\item {} 
\sphinxstyleliteralstrong{holdFlag} (\sphinxstyleliteralemphasis{bool}) -- DEFUNCT. NEEDS REMOVAL

\item {} 
\sphinxstyleliteralstrong{config} (\sphinxstyleliteralemphasis{str}) -- DEFUNCT. NEEDS REMOVAL

\end{itemize}

\item[{Returns}] \leavevmode
None

\end{description}\end{quote}

\end{fulllineitems}

\index{emr() (in module run)}

\begin{fulllineitems}
\phantomsection\label{\detokenize{run:run.emr}}\pysiglinewithargsret{\sphinxcode{run.}\sphinxbfcode{emr}}{\emph{calcs}, \emph{engine='`}, \emph{execPrefix=None}, \emph{execPostfix=None}, \emph{holdFlag=True}, \emph{config=None}}{}
Wrapper to set up GIPAW EMR calculation
\begin{quote}\begin{description}
\item[{Parameters}] \leavevmode
\sphinxstyleliteralstrong{calcs} (\sphinxstyleliteralemphasis{dict}) -- Dictionary of dictionaries of calculations

\item[{Keyword Arguments}] \leavevmode\begin{itemize}
\item {} 
\sphinxstyleliteralstrong{engine} (\sphinxstyleliteralemphasis{str}) -- executable that you are calling to run the calculations

\item {} 
\sphinxstyleliteralstrong{execPrefix} (\sphinxstyleliteralemphasis{str}) -- commands to go before the executable when run
(ex. mpiexec nice -n 19 \textless{}executable\textgreater{}) (default = None)

\item {} 
\sphinxstyleliteralstrong{execPostfix} (\sphinxstyleliteralemphasis{str}) -- commands to go after the executable when run
(ex. \textless{}execPrefix\textgreater{} \textless{}executable\textgreater{} -ndiag 12 -nimage 2) (default = None)

\item {} 
\sphinxstyleliteralstrong{holdFlag} (\sphinxstyleliteralemphasis{bool}) -- DEFUNCT. NEEDS REMOVAL

\item {} 
\sphinxstyleliteralstrong{config} (\sphinxstyleliteralemphasis{str}) -- DEFUNCT. NEEDS REMOVAL

\end{itemize}

\item[{Returns}] \leavevmode
None

\end{description}\end{quote}

\end{fulllineitems}

\index{generateSubRef() (in module run)}

\begin{fulllineitems}
\phantomsection\label{\detokenize{run:run.generateSubRef}}\pysiglinewithargsret{\sphinxcode{run.}\sphinxbfcode{generateSubRef}}{\emph{qsubRefFileString}, \emph{oneCalc}, \emph{ID}}{}
Reads in the reference cluster submission file specified in ``jobreffile''
in the config used. Tries to insert a few parameters.

OBSOLETE PLANNED FOR REMOVAL
\begin{quote}\begin{description}
\item[{Parameters}] \leavevmode\begin{itemize}
\item {} 
\sphinxstyleliteralstrong{qsubRefFileString} (\sphinxstyleliteralemphasis{str}) -- string of the ``reference'' cluster submission file

\item {} 
\sphinxstyleliteralstrong{oneCalc} (\sphinxstyleliteralemphasis{dict}) -- dictionary of one of the calculations

\item {} 
\sphinxstyleliteralstrong{ID} (\sphinxstyleliteralemphasis{str}) -- ID of calculation

\end{itemize}

\item[{Keyword Arguments}] \leavevmode
\sphinxstyleliteralstrong{None} -- 

\item[{Returns}] \leavevmode
string cluster submission file

\item[{Return type}] \leavevmode
clusterTypeDict (dict)

\end{description}\end{quote}

\end{fulllineitems}

\index{gvectors() (in module run)}

\begin{fulllineitems}
\phantomsection\label{\detokenize{run:run.gvectors}}\pysiglinewithargsret{\sphinxcode{run.}\sphinxbfcode{gvectors}}{\emph{calcs}, \emph{engine='`}, \emph{execPrefix=None}, \emph{execPostfix=None}, \emph{holdFlag=True}, \emph{config=None}}{}
Wrapper to set up GIPAW gvectors calculation
\begin{quote}\begin{description}
\item[{Parameters}] \leavevmode
\sphinxstyleliteralstrong{calcs} (\sphinxstyleliteralemphasis{dict}) -- Dictionary of dictionaries of calculations

\item[{Keyword Arguments}] \leavevmode\begin{itemize}
\item {} 
\sphinxstyleliteralstrong{engine} (\sphinxstyleliteralemphasis{str}) -- executable that you are calling to run the calculations

\item {} 
\sphinxstyleliteralstrong{execPrefix} (\sphinxstyleliteralemphasis{str}) -- commands to go before the executable when run
(ex. mpiexec nice -n 19 \textless{}executable\textgreater{}) (default = None)

\item {} 
\sphinxstyleliteralstrong{execPostfix} (\sphinxstyleliteralemphasis{str}) -- commands to go after the executable when run
(ex. \textless{}execPrefix\textgreater{} \textless{}executable\textgreater{} -ndiag 12 -nimage 2) (default = None)

\item {} 
\sphinxstyleliteralstrong{holdFlag} (\sphinxstyleliteralemphasis{bool}) -- DEFUNCT. NEEDS REMOVAL

\item {} 
\sphinxstyleliteralstrong{config} (\sphinxstyleliteralemphasis{str}) -- DEFUNCT. NEEDS REMOVAL

\end{itemize}

\item[{Returns}] \leavevmode
None

\end{description}\end{quote}

\end{fulllineitems}

\index{hyperfine() (in module run)}

\begin{fulllineitems}
\phantomsection\label{\detokenize{run:run.hyperfine}}\pysiglinewithargsret{\sphinxcode{run.}\sphinxbfcode{hyperfine}}{\emph{calcs}, \emph{engine='`}, \emph{execPrefix=None}, \emph{execPostfix=None}, \emph{holdFlag=True}, \emph{config=None}, \emph{isotope=()}}{}
Wrapper to set up GIPAW hyperfine calculation
\begin{quote}\begin{description}
\item[{Parameters}] \leavevmode
\sphinxstyleliteralstrong{calcs} (\sphinxstyleliteralemphasis{dict}) -- Dictionary of dictionaries of calculations

\item[{Keyword Arguments}] \leavevmode\begin{itemize}
\item {} 
\sphinxstyleliteralstrong{engine} (\sphinxstyleliteralemphasis{str}) -- executable that you are calling to run the calculations

\item {} 
\sphinxstyleliteralstrong{execPrefix} (\sphinxstyleliteralemphasis{str}) -- commands to go before the executable when run
(ex. mpiexec nice -n 19 \textless{}executable\textgreater{}) (default = None)

\item {} 
\sphinxstyleliteralstrong{execPostfix} (\sphinxstyleliteralemphasis{str}) -- commands to go after the executable when run
(ex. \textless{}execPrefix\textgreater{} \textless{}executable\textgreater{} -ndiag 12 -nimage 2) (default = None)

\item {} 
\sphinxstyleliteralstrong{holdFlag} (\sphinxstyleliteralemphasis{bool}) -- DEFUNCT. NEEDS REMOVAL

\item {} 
\sphinxstyleliteralstrong{config} (\sphinxstyleliteralemphasis{str}) -- DEFUNCT. NEEDS REMOVAL

\end{itemize}

\item[{Returns}] \leavevmode
None

\end{description}\end{quote}

\end{fulllineitems}

\index{nmr() (in module run)}

\begin{fulllineitems}
\phantomsection\label{\detokenize{run:run.nmr}}\pysiglinewithargsret{\sphinxcode{run.}\sphinxbfcode{nmr}}{\emph{calcs}, \emph{engine='`}, \emph{execPrefix=None}, \emph{execPostfix=None}, \emph{holdFlag=True}, \emph{config=None}}{}
Wrapper to set up GIPAW NMR calculation
\begin{quote}\begin{description}
\item[{Parameters}] \leavevmode
\sphinxstyleliteralstrong{calcs} (\sphinxstyleliteralemphasis{dict}) -- Dictionary of dictionaries of calculations

\item[{Keyword Arguments}] \leavevmode\begin{itemize}
\item {} 
\sphinxstyleliteralstrong{engine} (\sphinxstyleliteralemphasis{str}) -- executable that you are calling to run the calculations

\item {} 
\sphinxstyleliteralstrong{execPrefix} (\sphinxstyleliteralemphasis{str}) -- commands to go before the executable when run
(ex. mpiexec nice -n 19 \textless{}executable\textgreater{}) (default = None)

\item {} 
\sphinxstyleliteralstrong{execPostfix} (\sphinxstyleliteralemphasis{str}) -- commands to go after the executable when run
(ex. \textless{}execPrefix\textgreater{} \textless{}executable\textgreater{} -ndiag 12 -nimage 2) (default = None)

\item {} 
\sphinxstyleliteralstrong{holdFlag} (\sphinxstyleliteralemphasis{bool}) -- DEFUNCT. NEEDS REMOVAL

\item {} 
\sphinxstyleliteralstrong{config} (\sphinxstyleliteralemphasis{str}) -- DEFUNCT. NEEDS REMOVAL

\end{itemize}

\item[{Returns}] \leavevmode
None

\end{description}\end{quote}

\end{fulllineitems}

\index{pdos() (in module run)}

\begin{fulllineitems}
\phantomsection\label{\detokenize{run:run.pdos}}\pysiglinewithargsret{\sphinxcode{run.}\sphinxbfcode{pdos}}{\emph{calcs}, \emph{engine='`}, \emph{execPrefix=None}, \emph{execPostfix='`}, \emph{holdFlag=True}, \emph{config=None}}{}
Wrapper to set up DOS projection calculation
\begin{quote}\begin{description}
\item[{Parameters}] \leavevmode
\sphinxstyleliteralstrong{calcs} (\sphinxstyleliteralemphasis{dict}) -- Dictionary of dictionaries of calculations

\item[{Keyword Arguments}] \leavevmode\begin{itemize}
\item {} 
\sphinxstyleliteralstrong{engine} (\sphinxstyleliteralemphasis{str}) -- executable that you are calling to run the calculations

\item {} 
\sphinxstyleliteralstrong{execPrefix} (\sphinxstyleliteralemphasis{str}) -- commands to go before the executable when run
(ex. mpiexec nice -n 19 \textless{}executable\textgreater{}) (default = None)

\item {} 
\sphinxstyleliteralstrong{execPostfix} (\sphinxstyleliteralemphasis{str}) -- commands to go after the executable when run
(ex. \textless{}execPrefix\textgreater{} \textless{}executable\textgreater{} -ndiag 12 -nimage 2) (default = None)

\item {} 
\sphinxstyleliteralstrong{holdFlag} (\sphinxstyleliteralemphasis{bool}) -- DEFUNCT. NEEDS REMOVAL

\item {} 
\sphinxstyleliteralstrong{config} (\sphinxstyleliteralemphasis{str}) -- DEFUNCT. NEEDS REMOVAL

\end{itemize}

\item[{Returns}] \leavevmode
None

\end{description}\end{quote}

\end{fulllineitems}

\index{prep\_fd() (in module run)}

\begin{fulllineitems}
\phantomsection\label{\detokenize{run:run.prep_fd}}\pysiglinewithargsret{\sphinxcode{run.}\sphinxbfcode{prep\_fd}}{\emph{\_\_submitNodeName\_\_}, \emph{oneCalc}, \emph{ID}, \emph{nrx1=2}, \emph{nrx2=2}, \emph{nrx3=2}, \emph{innx=2}, \emph{de=0.01}, \emph{atom\_sym=True}, \emph{disp\_sym=True}, \emph{proj\_phDOS=True}}{}
Generates input files for fd.x, fd\_ifc.x, and matdyn.x for the finite
difference phonon calculations.
\begin{quote}\begin{description}
\item[{Parameters}] \leavevmode\begin{itemize}
\item {} 
\sphinxstyleliteralstrong{\_\_submitNodeName\_\_} (\sphinxstyleliteralemphasis{str}) -- String of hostname that cluster jobs should be submitted from

\item {} 
\sphinxstyleliteralstrong{oneCalc} (\sphinxstyleliteralemphasis{dict}) -- dictionary of one of the calculations

\item {} 
\sphinxstyleliteralstrong{ID} (\sphinxstyleliteralemphasis{str}) -- ID of calculation

\end{itemize}

\item[{Keyword Arguments}] \leavevmode\begin{itemize}
\item {} 
\sphinxstyleliteralstrong{nrx1} (\sphinxstyleliteralemphasis{int}) -- supercell size for first primitive lattice vector

\item {} 
\sphinxstyleliteralstrong{nrx2} (\sphinxstyleliteralemphasis{int}) -- supercell size for second primitive lattice vector

\item {} 
\sphinxstyleliteralstrong{nrx3} (\sphinxstyleliteralemphasis{int}) -- supercell size for third primitive lattice vector

\item {} 
\sphinxstyleliteralstrong{innx} (\sphinxstyleliteralemphasis{int}) -- how many differernt shifts in each direction for
finite difference phonon calculation

\item {} 
\sphinxstyleliteralstrong{de} (\sphinxstyleliteralemphasis{float}) -- amount to shift the atoms for finite differences

\end{itemize}

\item[{Returns}] \leavevmode
None

\end{description}\end{quote}

\end{fulllineitems}

\index{reduce\_kpoints() (in module run)}

\begin{fulllineitems}
\phantomsection\label{\detokenize{run:run.reduce_kpoints}}\pysiglinewithargsret{\sphinxcode{run.}\sphinxbfcode{reduce\_kpoints}}{\emph{inputfile}, \emph{factor}}{}
\end{fulllineitems}

\index{reset\_logs() (in module run)}

\begin{fulllineitems}
\phantomsection\label{\detokenize{run:run.reset_logs}}\pysiglinewithargsret{\sphinxcode{run.}\sphinxbfcode{reset\_logs}}{\emph{calcs}}{}
Removes log files from AFLOWpi directory
\begin{quote}\begin{description}
\item[{Parameters}] \leavevmode
\sphinxstyleliteralstrong{calcs} (\sphinxstyleliteralemphasis{dict}) -- Dictionary of dictionaries of calculations

\item[{Keyword Arguments}] \leavevmode
\sphinxstyleliteralstrong{None} -- 

\item[{Returns}] \leavevmode
None

\end{description}\end{quote}

\end{fulllineitems}

\index{resubmit() (in module run)}

\begin{fulllineitems}
\phantomsection\label{\detokenize{run:run.resubmit}}\pysiglinewithargsret{\sphinxcode{run.}\sphinxbfcode{resubmit}}{\emph{calcs}}{}
Stages loaded calculation set to be resubmitted on a cluster
\begin{quote}\begin{description}
\item[{Parameters}] \leavevmode
\sphinxstyleliteralstrong{calcs} (\sphinxstyleliteralemphasis{dict}) -- Dictionary of dictionaries of calculations

\item[{Keyword Arguments}] \leavevmode
\sphinxstyleliteralstrong{None} -- 

\item[{Returns}] \leavevmode
None

\end{description}\end{quote}

\end{fulllineitems}

\index{scf() (in module run)}

\begin{fulllineitems}
\phantomsection\label{\detokenize{run:run.scf}}\pysiglinewithargsret{\sphinxcode{run.}\sphinxbfcode{scf}}{\emph{calcs}, \emph{engine='`}, \emph{execPrefix=None}, \emph{execPostfix=None}, \emph{holdFlag=True}, \emph{config=None}, \emph{exit\_on\_error=True}}{}
Wrapper to set up self-consitent calculation
\begin{quote}\begin{description}
\item[{Parameters}] \leavevmode
\sphinxstyleliteralstrong{calcs} (\sphinxstyleliteralemphasis{dict}) -- Dictionary of dictionaries of calculations

\item[{Keyword Arguments}] \leavevmode\begin{itemize}
\item {} 
\sphinxstyleliteralstrong{engine} (\sphinxstyleliteralemphasis{str}) -- executable that you are calling to run the calculations

\item {} 
\sphinxstyleliteralstrong{execPrefix} (\sphinxstyleliteralemphasis{str}) -- commands to go before the executable when run
(ex. mpiexec nice -n 19 \textless{}executable\textgreater{}) (default = None)

\item {} 
\sphinxstyleliteralstrong{execPostfix} (\sphinxstyleliteralemphasis{str}) -- commands to go after the executable when run
(ex. \textless{}execPrefix\textgreater{} \textless{}executable\textgreater{} -ndiag 12 -nimage 2) (default = None)

\item {} 
\sphinxstyleliteralstrong{holdFlag} (\sphinxstyleliteralemphasis{bool}) -- DEFUNCT. NEEDS REMOVAL

\item {} 
\sphinxstyleliteralstrong{config} (\sphinxstyleliteralemphasis{str}) -- DEFUNCT. NEEDS REMOVAL

\end{itemize}

\end{description}\end{quote}

Returns:

\end{fulllineitems}

\index{submit() (in module run)}

\begin{fulllineitems}
\phantomsection\label{\detokenize{run:run.submit}}\pysiglinewithargsret{\sphinxcode{run.}\sphinxbfcode{submit}}{}{}
sets global \_\_submit\_\_flag\_\_ so calculations will start when user script completes
\begin{quote}\begin{description}
\item[{Parameters}] \leavevmode
\sphinxstyleliteralstrong{None} -- 

\item[{Keyword Arguments}] \leavevmode
\sphinxstyleliteralstrong{None} -- 

\item[{Returns}] \leavevmode
None

\end{description}\end{quote}

\end{fulllineitems}

\index{submitFirstCalcs\_\_() (in module run)}

\begin{fulllineitems}
\phantomsection\label{\detokenize{run:run.submitFirstCalcs__}}\pysiglinewithargsret{\sphinxcode{run.}\sphinxbfcode{submitFirstCalcs\_\_}}{\emph{calcs}}{}
Submits the first step of a calculation's pipeline
\begin{quote}\begin{description}
\item[{Parameters}] \leavevmode
\sphinxstyleliteralstrong{calcs} (\sphinxstyleliteralemphasis{dict}) -- Dictionary of dictionaries of calculations

\item[{Keyword Arguments}] \leavevmode
\sphinxstyleliteralstrong{None} -- 

\item[{Returns}] \leavevmode
None

\end{description}\end{quote}

\end{fulllineitems}

\index{testOne() (in module run)}

\begin{fulllineitems}
\phantomsection\label{\detokenize{run:run.testOne}}\pysiglinewithargsret{\sphinxcode{run.}\sphinxbfcode{testOne}}{\emph{calcs}, \emph{calcType='scf'}, \emph{engine='`}, \emph{execPrefix=None}, \emph{execPostfix=None}, \emph{holdFlag=True}, \emph{config=None}}{}
Run all the calculation in the dictionary with a specific engine
\begin{quote}\begin{description}
\item[{Parameters}] \leavevmode
\sphinxstyleliteralstrong{calcs} (\sphinxstyleliteralemphasis{dict}) -- Dictionary of dictionaries of calculations

\item[{Keyword Arguments}] \leavevmode\begin{itemize}
\item {} 
\sphinxstyleliteralstrong{engine} (\sphinxstyleliteralemphasis{str}) -- executable that you are calling to run the calculations

\item {} 
\sphinxstyleliteralstrong{execPrefix} (\sphinxstyleliteralemphasis{str}) -- commands to go before the executable when run
(ex. mpiexec nice -n 19 \textless{}executable\textgreater{}) (default = None)

\item {} 
\sphinxstyleliteralstrong{execPostfix} (\sphinxstyleliteralemphasis{str}) -- commands to go after the executable when run
(ex. \textless{}execPrefix\textgreater{} \textless{}executable\textgreater{} -ndiag 12 -nimage 2) (default = None)

\end{itemize}

\item[{Returns}] \leavevmode
None

\end{description}\end{quote}

\end{fulllineitems}

\index{write\_fdx\_template() (in module run)}

\begin{fulllineitems}
\phantomsection\label{\detokenize{run:run.write_fdx_template}}\pysiglinewithargsret{\sphinxcode{run.}\sphinxbfcode{write\_fdx\_template}}{\emph{oneCalc}, \emph{ID}, \emph{nrx1=2}, \emph{nrx2=2}, \emph{nrx3=2}, \emph{innx=2}, \emph{de=0.01}, \emph{atom\_sym=True}, \emph{disp\_sym=True}, \emph{proj\_phDOS=True}}{}
Generates input files for fd.x, fd\_ifc.x, and matdyn.x for the finite
difference phonon calculations.
\begin{quote}\begin{description}
\item[{Parameters}] \leavevmode\begin{itemize}
\item {} 
\sphinxstyleliteralstrong{oneCalc} (\sphinxstyleliteralemphasis{dict}) -- dictionary of one of the calculations

\item {} 
\sphinxstyleliteralstrong{ID} (\sphinxstyleliteralemphasis{str}) -- ID of calculation

\end{itemize}

\item[{Keyword Arguments}] \leavevmode\begin{itemize}
\item {} 
\sphinxstyleliteralstrong{nrx1} (\sphinxstyleliteralemphasis{int}) -- supercell size for first primitive lattice vector

\item {} 
\sphinxstyleliteralstrong{nrx2} (\sphinxstyleliteralemphasis{int}) -- supercell size for second primitive lattice vector

\item {} 
\sphinxstyleliteralstrong{nrx3} (\sphinxstyleliteralemphasis{int}) -- supercell size for third primitive lattice vector

\item {} 
\sphinxstyleliteralstrong{innx} (\sphinxstyleliteralemphasis{int}) -- how many differernt shifts in each direction for
finite difference phonon calculation

\item {} 
\sphinxstyleliteralstrong{de} (\sphinxstyleliteralemphasis{float}) -- amount to shift the atoms for finite differences

\end{itemize}

\item[{Returns}] \leavevmode
None

\end{description}\end{quote}

\end{fulllineitems}



\chapter{scfuj module}
\label{\detokenize{scfuj:module-scfuj}}\label{\detokenize{scfuj::doc}}\label{\detokenize{scfuj:scfuj-module}}\index{scfuj (module)}\index{WanT\_bands() (in module scfuj)}

\begin{fulllineitems}
\phantomsection\label{\detokenize{scfuj:scfuj.WanT_bands}}\pysiglinewithargsret{\sphinxcode{scfuj.}\sphinxbfcode{WanT\_bands}}{\emph{oneCalc}, \emph{ID=None}, \emph{eShift=5.5}, \emph{num\_points=1000}, \emph{cond\_bands=True}, \emph{compute\_ham=False}, \emph{proj\_thr=0.9}}{}
Make input files for  WanT bands calculation
\begin{quote}\begin{description}
\item[{Parameters}] \leavevmode
\sphinxstyleliteralstrong{calc\_copy -{-} dictionary of dictionaries of calculations} (\sphinxstyleliteralemphasis{-}) -- 

\end{description}\end{quote}

\end{fulllineitems}

\index{WanT\_dos() (in module scfuj)}

\begin{fulllineitems}
\phantomsection\label{\detokenize{scfuj:scfuj.WanT_dos}}\pysiglinewithargsret{\sphinxcode{scfuj.}\sphinxbfcode{WanT\_dos}}{\emph{oneCalc, ID=None, eShift=5.0, temperature=None, energy\_range={[}-21.0, 21.0{]}, boltzmann=True, k\_grid=None, pdos=False, num\_e=4001, cond\_bands=True, fermi\_surface=False, compute\_ham=False, proj\_thr=0.95}}{}
Make input files for  WanT bands calculation
\begin{quote}\begin{description}
\item[{Parameters}] \leavevmode
\sphinxstyleliteralstrong{calc\_copy -{-} dictionary of dictionaries of calculations} (\sphinxstyleliteralemphasis{-}) -- 

\end{description}\end{quote}

\end{fulllineitems}

\index{WanT\_epsilon() (in module scfuj)}

\begin{fulllineitems}
\phantomsection\label{\detokenize{scfuj:scfuj.WanT_epsilon}}\pysiglinewithargsret{\sphinxcode{scfuj.}\sphinxbfcode{WanT\_epsilon}}{\emph{oneCalc, ID=None, eShift=5.0, temperature=300.0, energy\_range={[}0.01, 8.01{]}, ne=160, k\_grid=None, compute\_ham=False, proj\_thr=0.95, proj\_nbnd=None}}{}
Make input files for  WanT bands calculation
\begin{quote}\begin{description}
\item[{Parameters}] \leavevmode
\sphinxstyleliteralstrong{calc\_copy -{-} dictionary of dictionaries of calculations} (\sphinxstyleliteralemphasis{-}) -- 

\end{description}\end{quote}

\end{fulllineitems}

\index{acbn0() (in module scfuj)}

\begin{fulllineitems}
\phantomsection\label{\detokenize{scfuj:scfuj.acbn0}}\pysiglinewithargsret{\sphinxcode{scfuj.}\sphinxbfcode{acbn0}}{\emph{oneCalc}, \emph{projCalcID}, \emph{byAtom=False}}{}
\end{fulllineitems}

\index{checkOscillation() (in module scfuj)}

\begin{fulllineitems}
\phantomsection\label{\detokenize{scfuj:scfuj.checkOscillation}}\pysiglinewithargsret{\sphinxcode{scfuj.}\sphinxbfcode{checkOscillation}}{\emph{ID}, \emph{oneCalc}, \emph{uThresh=0.001}}{}
\end{fulllineitems}

\index{chkSpinCalc() (in module scfuj)}

\begin{fulllineitems}
\phantomsection\label{\detokenize{scfuj:scfuj.chkSpinCalc}}\pysiglinewithargsret{\sphinxcode{scfuj.}\sphinxbfcode{chkSpinCalc}}{\emph{oneCalc}, \emph{ID=None}}{}
Check whether an calculation is spin polarized or not.

Arguments:

--oneCalc : dictionary of a single calculation.

\end{fulllineitems}

\index{chk\_species() (in module scfuj)}

\begin{fulllineitems}
\phantomsection\label{\detokenize{scfuj:scfuj.chk_species}}\pysiglinewithargsret{\sphinxcode{scfuj.}\sphinxbfcode{chk\_species}}{\emph{elm}}{}
\end{fulllineitems}

\index{evCurveMinimize() (in module scfuj)}

\begin{fulllineitems}
\phantomsection\label{\detokenize{scfuj:scfuj.evCurveMinimize}}\pysiglinewithargsret{\sphinxcode{scfuj.}\sphinxbfcode{evCurveMinimize}}{\emph{calcs}, \emph{config=None}, \emph{pThresh=10.0}, \emph{final\_minimization='vc-relax'}}{}
\end{fulllineitems}

\index{getU\_frmACBN0out() (in module scfuj)}

\begin{fulllineitems}
\phantomsection\label{\detokenize{scfuj:scfuj.getU_frmACBN0out}}\pysiglinewithargsret{\sphinxcode{scfuj.}\sphinxbfcode{getU\_frmACBN0out}}{\emph{oneCalc}, \emph{ID}, \emph{byAtom=False}}{}
\end{fulllineitems}

\index{maketree() (in module scfuj)}

\begin{fulllineitems}
\phantomsection\label{\detokenize{scfuj:scfuj.maketree}}\pysiglinewithargsret{\sphinxcode{scfuj.}\sphinxbfcode{maketree}}{\emph{oneCalc}, \emph{ID}, \emph{paodir=None}}{}
Make the directoy tree and place in the input file there
\begin{quote}\begin{description}
\item[{Parameters}] \leavevmode
\sphinxstyleliteralstrong{calcs -{-} Dictionary of dictionaries of calculations} (\sphinxstyleliteralemphasis{-}) -- 

\item[{Keyword Arguments}] \leavevmode\begin{itemize}
\item {} 
\sphinxstyleliteralstrong{pseudodir   -{-} path of pseudopotential files directory} (\sphinxstyleliteralemphasis{-}) -- 

\item {} 
\sphinxstyleliteralstrong{paodir       - path of pseudoatomic orbital basis set} (\sphinxstyleliteralemphasis{-}) -- 

\end{itemize}

\end{description}\end{quote}

\end{fulllineitems}

\index{nscf\_nosym\_noinv() (in module scfuj)}

\begin{fulllineitems}
\phantomsection\label{\detokenize{scfuj:scfuj.nscf_nosym_noinv}}\pysiglinewithargsret{\sphinxcode{scfuj.}\sphinxbfcode{nscf\_nosym\_noinv}}{\emph{oneCalc}, \emph{ID=None}, \emph{kpFactor=1.5}, \emph{unoccupied\_states=False}}{}
Add the ncsf input to each subdir and update the master dictionary
\begin{quote}\begin{description}
\item[{Parameters}] \leavevmode\begin{itemize}
\item {} 
\sphinxstyleliteralstrong{calc\_copy -{-} dictionary of one calculation} (\sphinxstyleliteralemphasis{-}) -- 

\item {} 
\sphinxstyleliteralstrong{kpFactor -{-} multiplicative factor for kpoints from SCF to DOS}\sphinxstyleliteralstrong{ (}\sphinxstyleliteralstrong{default} (\sphinxstyleliteralemphasis{-}) -- \begin{enumerate}
\setcounter{enumi}{1}
\item {} 
\end{enumerate}


\end{itemize}

\end{description}\end{quote}

\end{fulllineitems}

\index{projwfc() (in module scfuj)}

\begin{fulllineitems}
\phantomsection\label{\detokenize{scfuj:scfuj.projwfc}}\pysiglinewithargsret{\sphinxcode{scfuj.}\sphinxbfcode{projwfc}}{\emph{oneCalc}, \emph{ID=None}, \emph{paw=False}}{}
Run projwfc on each calculation
\begin{quote}\begin{description}
\item[{Parameters}] \leavevmode
\sphinxstyleliteralstrong{oneCalc -{-} dictionary of a single calculation} (\sphinxstyleliteralemphasis{-}) -- 

\end{description}\end{quote}

\end{fulllineitems}

\index{run() (in module scfuj)}

\begin{fulllineitems}
\phantomsection\label{\detokenize{scfuj:scfuj.run}}\pysiglinewithargsret{\sphinxcode{scfuj.}\sphinxbfcode{run}}{\emph{calcs}, \emph{uThresh=0.001}, \emph{nIters=20}, \emph{mixing=0.7}, \emph{kp\_mult=1.5}}{}
\end{fulllineitems}

\index{run\_transport() (in module scfuj)}

\begin{fulllineitems}
\phantomsection\label{\detokenize{scfuj:scfuj.run_transport}}\pysiglinewithargsret{\sphinxcode{scfuj.}\sphinxbfcode{run\_transport}}{\emph{\_\_submitNodeName\_\_, oneCalc, ID, run\_scf=True, run\_transport\_prep=True, run\_bands=False, epsilon=False, temperature=300, en\_range={[}0.05, 10.0{]}, ne=1000, compute\_ham=False, proj\_thr=0.95, proj\_sh=5.5, proj\_nbnd=True}}{}
\end{fulllineitems}

\index{scfprep() (in module scfuj)}

\begin{fulllineitems}
\phantomsection\label{\detokenize{scfuj:scfuj.scfprep}}\pysiglinewithargsret{\sphinxcode{scfuj.}\sphinxbfcode{scfprep}}{\emph{calcs}, \emph{paodir=None}}{}
\end{fulllineitems}

\index{tb\_prep() (in module scfuj)}

\begin{fulllineitems}
\phantomsection\label{\detokenize{scfuj:scfuj.tb_prep}}\pysiglinewithargsret{\sphinxcode{scfuj.}\sphinxbfcode{tb\_prep}}{\emph{oneCalc}, \emph{ID}}{}
\end{fulllineitems}

\index{transport\_prep() (in module scfuj)}

\begin{fulllineitems}
\phantomsection\label{\detokenize{scfuj:scfuj.transport_prep}}\pysiglinewithargsret{\sphinxcode{scfuj.}\sphinxbfcode{transport\_prep}}{\emph{oneCalc}, \emph{ID}}{}
sets up the environment do do scf-\textgreater{}nscf-\textgreater{}projwfc to get overlap for transport calcs

\end{fulllineitems}

\index{updateUvals() (in module scfuj)}

\begin{fulllineitems}
\phantomsection\label{\detokenize{scfuj:scfuj.updateUvals}}\pysiglinewithargsret{\sphinxcode{scfuj.}\sphinxbfcode{updateUvals}}{\emph{oneCalc}, \emph{Uvals}, \emph{ID=None}}{}
Modify scf input file to do a lda+u calculation.
\begin{quote}\begin{description}
\item[{Parameters}] \leavevmode\begin{itemize}
\item {} 
\sphinxstyleliteralstrong{oneCalc      -{-} Dictionary of one calculation} (\sphinxstyleliteralemphasis{-}) -- 

\item {} 
\sphinxstyleliteralstrong{Uvals        -{-} Dictionary of Uvals} (\sphinxstyleliteralemphasis{-}) -- 

\end{itemize}

\end{description}\end{quote}

\end{fulllineitems}

\index{want\_bands\_prep() (in module scfuj)}

\begin{fulllineitems}
\phantomsection\label{\detokenize{scfuj:scfuj.want_bands_prep}}\pysiglinewithargsret{\sphinxcode{scfuj.}\sphinxbfcode{want\_bands\_prep}}{\emph{oneCalc}, \emph{ID}}{}
\end{fulllineitems}

\index{want\_dos\_prep() (in module scfuj)}

\begin{fulllineitems}
\phantomsection\label{\detokenize{scfuj:scfuj.want_dos_prep}}\pysiglinewithargsret{\sphinxcode{scfuj.}\sphinxbfcode{want\_dos\_prep}}{\emph{oneCalc}, \emph{ID}}{}
\end{fulllineitems}

\index{want\_eff\_mass\_prep() (in module scfuj)}

\begin{fulllineitems}
\phantomsection\label{\detokenize{scfuj:scfuj.want_eff_mass_prep}}\pysiglinewithargsret{\sphinxcode{scfuj.}\sphinxbfcode{want\_eff\_mass\_prep}}{\emph{oneCalc}, \emph{ID}}{}
\end{fulllineitems}

\index{want\_epsilon\_prep() (in module scfuj)}

\begin{fulllineitems}
\phantomsection\label{\detokenize{scfuj:scfuj.want_epsilon_prep}}\pysiglinewithargsret{\sphinxcode{scfuj.}\sphinxbfcode{want\_epsilon\_prep}}{\emph{oneCalc, ID, en\_range={[}0.5, 10.0{]}, ne=95}}{}
\end{fulllineitems}



\chapter{Indices and tables}
\label{\detokenize{index:indices-and-tables}}\begin{itemize}
\item {} 
\DUrole{xref,std,std-ref}{genindex}

\item {} 
\DUrole{xref,std,std-ref}{modindex}

\item {} 
\DUrole{xref,std,std-ref}{search}

\end{itemize}


\renewcommand{\indexname}{Python Module Index}
\begin{sphinxtheindex}
\def\bigletter#1{{\Large\sffamily#1}\nopagebreak\vspace{1mm}}
\bigletter{p}
\item {\sphinxstyleindexentry{plot}}\sphinxstyleindexpageref{plot:\detokenize{module-plot}}
\item {\sphinxstyleindexentry{prep}}\sphinxstyleindexpageref{prep:\detokenize{module-prep}}
\item {\sphinxstyleindexentry{pseudo}}\sphinxstyleindexpageref{pseudo:\detokenize{module-pseudo}}
\indexspace
\bigletter{r}
\item {\sphinxstyleindexentry{retr}}\sphinxstyleindexpageref{retr:\detokenize{module-retr}}
\item {\sphinxstyleindexentry{run}}\sphinxstyleindexpageref{run:\detokenize{module-run}}
\indexspace
\bigletter{s}
\item {\sphinxstyleindexentry{scfuj}}\sphinxstyleindexpageref{scfuj:\detokenize{module-scfuj}}
\end{sphinxtheindex}

\renewcommand{\indexname}{Index}
\printindex
\end{document}